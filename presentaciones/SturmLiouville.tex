

% Inicio de la presentacion
% Para ejecutar el Python
% /usr/share/texlive/texmf-dist/scripts/pythontex/pythontex3.py EcuacionesLineales.tex
%\documentclass[handout]{beamer}
\documentclass[xcolor=dvipsnames,a4paper,10pt,handout]{beamer}
	\mode<presentation>
\usetheme{Warsaw}
\usecolortheme[named=Salmon]{structure}%rosa


\newcommand{\rr}{\mathbb{R}}
\renewcommand{\emph}[1]{\textcolor{blue}{#1}}
\renewcommand{\textbf}[1]{\textcolor{green}{\bfseries #1}}



\setbeamertemplate{footline}
{
  \hbox{\begin{beamercolorbox}[wd=1\paperwidth,ht=2.25ex,dp=1ex,right]{framenumber}%
      \usebeamerfont{framenumber}\insertframenumber{} / \inserttotalframenumber\hspace*{2ex}
    \end{beamercolorbox}}%
  \vskip0pt%
}
%\definecolor{beamer@blendedblue}{rgb}{0.137,0.466,0.741}
\definecolor{gris}{HTML}{7adcd2}
\definecolor{rojo}{HTML}{f2c08f}
\definecolor{frametit}{HTML}{fd779d}
% 
% \setbeamercolor{structure}{fg=beamer@blendedblue}
% \setbeamercolor{titlelike}{parent=structure}
% \setbeamercolor{frametitle}{fg=black}
% \setbeamercolor{title}{fg=black}
% \setbeamercolor{item}{fg=black}

 


%\usecolortheme{}
%\usecolortheme[named=Aquamarine]{structure}
%\usetheme{Singapore}%{Marburg}%{Berkeley}%{Antibes}%{Darmstadt}
%\usefonttheme{serif}


% 
% \setbeamertemplate{navigation symbols}{}
% \setbeamertemplate{footline}{}
% 
% Personalizar el color de los títulos
% 
% \setbeamercolor{title}{fg=black,bg=Aquamarine}
% \setbeamercolor{block title example}{fg=black,bg=Lavender!90}
% \setbeamercolor{block title alerted}{fg=black,bg=Lavender!80}
% \setbeamercolor{block body alerted}{fg=black,bg=Lavender!20}






\usepackage{tikz}
\usepackage{mathrsfs}
\usepackage[utf8]{inputenc}
\usepackage[spanish]{babel}
\usepackage{beamerthemesplit}
\usepackage{verbatim}
\usepackage{amssymb}
\usepackage{graphicx}
\usepackage{animate}
\usepackage[framemethod=TikZ]{mdframed}
\usepackage{empheq}
\usepackage[breakable,many]{tcolorbox}
\usepackage[beamer,customcolors]{hf-tikz}
\usepackage{hyperref}
\tcbset{nobeforeafter}

\tcbset{highlight math style={enhanced,%<-- needed for the `remember' options
colframe=frametit!100,blue!32,colback=frametit!100,boxsep=0pt}}

\pgfdeclareverticalshading{exersicebackground}{100bp}
  {color(0bp)=(blue!20);color(50bp)=(blue!0)}
\mdfdefinestyle{MiEstilo}{innertopmargin=10pt,linecolor=white!100,%
linewidth=2pt,topline=true,tikzsetting={shading=exersicebackground}}  
% 
% \usetikzlibrary{backgrounds}
% \usetikzlibrary{mindmap,trees}	
% 
% 
% \tikzstyle{every picture}+=[remember picture]

\usepackage{beamerhighlight}


%% libraries for snake paths
\usetikzlibrary{decorations}
\usetikzlibrary{decorations.pathmorphing}

%% library for star node
\usetikzlibrary{shapes.geometric}

%% library for computation of coordinates
\usetikzlibrary{calc}


%  
% %%%%%%%%%%%%%%%%%%%%%%%%%%Nuevos comandos entornos%%%%%%%%%%%%%%%%%%%%%%%%%%%%%%%%
% %%%%%%%%%%%%%%%%%%%%%%%%%%%%%%%%%%%%%%%%%%%%%%%%%%%%%%%%%%%%%%%%%%%%%%%%
\newenvironment{demo}{\noindent\emph{Dem.}}{\hfill\qed \newline\vspace{5pt}}

% \newenvironment{observa}{\noindent\textbf{Observación:}}{}
\newcommand{\com}{\mathbb{C}}
\newcommand{\rr}{\mathbb{R}}
\newcommand{\nn}{\mathbb{N}}
% \renewcommand{\epsilon}{\varepsilon}
\renewcommand{\lim}{\mathop{\rm lím}}
\renewcommand{\inf}{\mathop{\rm ínf}}
\renewcommand{\liminf}{\mathop{\rm líminf}}
\renewcommand{\limsup}{\mathop{\rm límsup}}
\renewcommand{\min}{\mathop{\rm mín}}
\renewcommand{\max}{\mathop{\rm máx}}
\renewcommand{\b}[1]{\boldsymbol{#1}}
% \renewenvironment{frame}[1]{}{}

%%%%%%%%%%%%%%%% Funcion característica %%%%%%%%%%%5555555

\DeclareRobustCommand{\rchi}{{\mathpalette\irchi\relax}}
\newcommand{\irchi}[2]{\raisebox{\depth}{$#1\chi$}} % inner command, used by \rchi
\newcommand{\der}[2]{\frac{\partial #1}{\partial #2}} 
 
%  \definecolor{mycolor}{RGB}{204,179,174}
% 
% \tcbset{highlight math style={enhanced,
%   colframe=red!60!black,colback=mycolor,arc=4pt,boxrule=1pt,
%   drop fuzzy shadow}}

% 
% 
% 


%\renewcommand{\lim}{displaystyle\lim}
\DeclareMathOperator{\atan2}{atan2}
\DeclareMathOperator{\sen}{sen}

\pgfdeclareverticalshading{exersicebackground}{100bp}
  {color(0bp)=(black!40);color(50bp)=(black!0)}

\mdfdefinestyle{MiEstilo}{innertopmargin=10pt,linecolor=white!100,%
linewidth=2pt,topline=true,tikzsetting={shading=exersicebackground}}  



%%%%%%%%%%%%%%%%  Recuadro ecuacion %%%%%%%%%%%%%%%%%%%%%%%%


\newcommand{\boxedeq}[2]{%
\begin{empheq}[box=\tcbhighmath]{equation}\label{#2} #1 \end{empheq}}

%\newcommand{boxedeq}[1]{\textbf{#1}}








%%%%%%%%%%%%%%%%%%%%%%%%%%%%%

%%%%%%%%%%%%%%%%%%%%%%%%%%%%%%
%% Codigo
\newenvironment{codigo}[1][]{%
\mdfsetup{style=MiEstilo%
}
\ifstrempty{#1}
{
\begin{mdframed}[]\relax%
\strut \textbf{Codigo}
}
{
\begin{mdframed}[]\relax%
\strut \textbf{Codigo (#1)}
}}{\end{mdframed}}
%%%%%%%%%%%%%%%%%%%%%%%%%%%%%




%%%%%%%%%%%%%%%%%%%%%Colores

\definecolor{color8}{HTML}{8E87C1}
\definecolor{color2}{rgb}{0.44,0.62,0.42}
\definecolor{color3}{rgb}{0.28, 0.51, .68}
\definecolor{color4}{rgb}{0.29,0.3,0.57}


%%%%%%%%%%%%%%%%%%%%%% Configuracion listing

\lstset{ %
  backgroundcolor=\color{white},   % choose the background color; you must add \usepackage{color} or \usepackage{xcolor}
  basicstyle=\footnotesize,        % the size of the fonts that are used for the code
  breakatwhitespace=false,         % sets if automatic breaks should only happen at whitespace
  breaklines=true,                 % sets automatic line breaking
  captionpos=b,                    % sets the caption-position to bottom
  commentstyle=\color{color2},    % comment style
  deletekeywords={...},            % if you want to delete keywords from the given language
  escapeinside={\%*}{*)},          % if you want to add LaTeX within your code
  extendedchars=true,              % lets you use non-ASCII characters; for 8-bits encodings only, does not work with UTF-8
  frame=single,	                   % adds a frame around the code
  keepspaces=true,                 % keeps spaces in text, useful for keeping indentation of code (possibly needs columns=flexible)
  keywordstyle=\color{blue},       % keyword style
  language=Python,                 % the language of the code
  otherkeywords={symbols,dsolve,solve,Eq, simplify, subs, plot,Function},           % if you want to add more keywords to the set
  numbers=left,                    % where to put the line-numbers; possible values are (none, left, right)
  numbersep=5pt,                   % how far the line-numbers are from the code
  numberstyle=\tiny\color{color3}, % the style that is used for the line-numbers
  rulecolor=\color{black},         % if not set, the frame-color may be changed on line-breaks within not-black text (e.g. comments (green here))
  showspaces=false,                % show spaces everywhere adding particular underscores; it overrides 'showstringspaces'
  showstringspaces=false,          % underline spaces within strings only
  showtabs=false,                  % show tabs within strings adding particular underscores
  stepnumber=2,                    % the step between two line-numbers. If it's 1, each line will be numbered
  stringstyle=\color{color4},     % string literal style
  tabsize=2,	                   % sets default tabsize to 2 spaces
  title=\lstname                   % show the filename of files included with \lstinputlisting; also try caption instead of title
}


% newcommand
%   
% \newcommand{\xb}{\vec{x}}
% \renewcommand{\emph}[1]{\textcolor{blue}{\bfseries #1}}
%\graphicspath{/home/fernando/fer/Docencia/posgrado/Derivadas Parciales Maestría/2022/unidad2-7}

 
\DeclareMathOperator{\sen}{sen}

\DeclareMathOperator{\opL}{\mathscr{L}}
\DeclareMathOperator{\opM}{\mathscr{M}}
\DeclareMathOperator{\dive}{div}

\title[Ecuaciones en Derivas Parciales-Aplicaciones] % (optional, nur bei langen Titeln nötig)
{%
Problemas de Sturm-Liouville
}

\author[] % (optional, nur bei vielen Autoren)
{ }

\institute[Problemas de Sturm-Liouville] % (optional, aber oft nötig)
{
 Depto de Matemática\\
Facultad de Ciencias Exactas Físico-Químicas y Naturales\\
Universidad Nacional de Río Cuarto}


\subject{Ecuaciones en Derivadas Parciales}


% 





% 
\begin{document}
% 

\hfsetfillcolor{frametit}
\hfsetbordercolor{frametit}

 

\begin{frame}
  \maketitle
  \begin{center}
    \includegraphics[scale=0.2]{/home/fernando/fer/Docencia/grado/EcuacionesDiferenciales/2012/unrc.jpg}
   \end{center}
\end{frame}

\begin{frame}{Bibliografía}

\begin{tabular}{ccc}
\includegraphics[scale=.15]{/home/fernando/fer/Biblioteca/calibre/R. Kent Nagle/Ecuaciones diferenciales y problemas con valores en la frontera (1305)/cover.jpg}
&
\includegraphics[scale=.14]{/home/fernando/fer/Biblioteca/calibre/Charles Henry Edwards/Ecuaciones diferenciales y problemas con valores en la frontera (1306)/cover.jpg}
&
\includegraphics[scale=.3]{/home/fernando/fer/Biblioteca/calibre/Garrett Birkhoff/Ordinary Differential Equations (1014)/cover.jpg}
\end{tabular}



\end{frame}




\section{Problemas físicos que llevan a problemas de Sturm-Liouville}
\begin{frame}{Ecuación del Calor}

Recordemos la ecuación diferencial del Calor 

\[ \frac{\partial c\rho u}{\partial t}-\operatorname{div}( k \nabla u)=S \]
\onslide<+->
con:

\begin{itemize}
\item<+-> $u$ temperatura del medio
 \item<+-> $c$  calor específico
 \item<+-> $\rho$ densidad
 \item<+-> $k$ coeficiente conductividad térmica
 \item<+-> $S$ fuente externa de calor
\end{itemize}
\onslide<+->
$c, \rho, k, S$ son funciones del tiempo $t$ y el espacio $(x,y,z)$. 
\end{frame}


\begin{frame}{Ley Enfriamiento de Newton}
\onslide<+->

En algunos problemas la fuente externa $S$  además de contener términos $h(x,y,y,t)$ que dependen de la posición y el tiempo contiene otros que dependen de  la temperatura $u$. 
\onslide<+->

Por ejemplo en la \emph{Ley Enfriamiento de Newton} el calor que ingresa a un cuerpo  es proporcional a la diferencia de temperatura entre el cuerpo y el medio circundante. 
\onslide<+->

 La ecuación se transforma

 \begin{empheq}[box=\tcbhighmath]{equation}\label{eq:calor+newton}
 \frac{\partial c\rho u}{\partial t}- \dive k \nabla u - q u=h.
\end{empheq}

También $q$ podría ser función de $t$ y $(x,y,z)$.
\end{frame}


\begin{frame}{Ecuación del calor uni-dimensional}

Para mayor simplicidad nos restringiremos al caso de un alambre recto y tan delgado que la asumimos uni-dimensional. La única variable espacial es $x\in [a,b]$ y $t>0$.

{\small
 \begin{empheq}[box=\tcbhighmath]{equation}\label{eq:calor-uni} 
 \quad c(x) \rho(x) \frac{\partial u}{\partial t}(x, t)=\frac{\partial}{\partial x}\left[k(x) \frac{\partial u}{\partial x}(x, t)\right]+q(x) u(x, t)+h(x, t).
 \end{empheq}
 }
 
 \end{frame}


\begin{frame}{Condiciones de contorno e iniciales}

\textbf{Condiciones de contorno: en $x=a$ o $x=b$}

\begin{description}
 \item<+->[ Extremos fijos (Dirichlet)] $u=0$
 \item<+->[Alambre aislado (Neuman)]   $\partial u / \partial x=0$
 \item<+->[Condiciones mixtas] $\partial u / \partial x+c u=0$.
\end{description}
\onslide<+->
Para generalizar la situación, supondremos

\begin{empheq}[box=\tcbhighmath]{equation}\label{eq:cond_cont}
\left\{\begin{array}{cc}
 a_{1} u(a, t)+a_{2} \frac{\partial u}{\partial x}(a, t)&=0\\ b_{1} u(b, t)+b_{2} \frac{\partial u}{\partial x}(b, t)&=0
\end{array}
\right.\end{empheq}
  
\textbf{Condiciones iniciales}
$$u(x,0)=f(x)$$
   
 \end{frame}


\begin{frame}{Problema}

Para simplificar nuestro problema supondremos que $h(x, t) \equiv 0$. 
Vamos a estudiar el siguiente problema de contorno y valores iniciales.



{\small
\begin{empheq}[box=\tcbhighmath]{equation}\label{eq:calor_main}  
\left\{
        \begin{array}{ll}
            r(x) \frac{\partial u}{\partial t}(x, t)=\frac{\partial}{\partial x}\left[p(x) \frac{\partial u}{\partial x}(x, t)\right]+q(x) u(x, t) &   a<x<b,  t>0,\\
            & \\
             a_{1} u(a, t)+a_{2} \frac{\partial u}{\partial x}(a, t)=0 &\  t>0,\\
             &\\
              b_{1} u(b, t)+b_{2} \frac{\partial u}{\partial x}(b, t)=0 & t>0,\\
              &\\
            u(x, 0)=f(x)&   a<x<b.
        \end{array}
\right. \notag
\end{empheq}
}

donde $r(x)=c(x) \rho(x)$ y $p(x)=k(x)$.

   
 \end{frame}


\begin{frame}{Problemas Sturm-Liouville}

\begin{block}{Ejercicio [Separación Variables]}
 
Supongamos $u(x, t)=y(x) T(t)$ resuelve la ecuación,  reemplazando en la ecuación principal demostrar  que existe $\lambda\in\rr$ (es un número por determinar)
\[
 \left\{
        \begin{array}{ll}
        \frac{d}{d x}\left[p(x) \frac{d y}{d x}\right]+q(x) y+\lambda r(x) y=0 & a<x<b,\\
        a_{1} y(a)+a_{2} y^{\prime}(a)=0, &\\
        b_{1} y(b)+b_{2} y^{\prime}(b)=0 & \\
        \end{array}
 \right.
 \]
\end{block}

Este sistema se llama  un \emph{problema de Sturm-Liouville} con valores en la frontera. 

Excluimos las condiciones triviales en la frontera, donde $a_{1}=a_{2}=0$ o $b_{1}=b_{2}=0$. 



 \end{frame}
 

\subsection{Clasificación}
\begin{frame}{Clasificación problemas Sturm-Liouville}
\begin{block}{Definición}
 Cuando:
 \begin{itemize}
  \item  $p(x), q(x)$ y $r(x)$ son continuas en $[a, b]$, 
  \item $p'$ derivable en $(a,b)$,
  \item $p(x)>0$ y $r(x)>0$ en $[a, b]$, 
 \end{itemize}
decimos que tenemos un  \emph{ problema regular de Sturm-Liouville}.
\onslide<+->

Decimos que la ecuación es \emph{singular} si:
 \begin{itemize}
  \item $p$ se anula en $a$ o  $ b$,
  \item si $p(x), q(x)$ o $r(x)$ no están acotadas cuando $x$ tiende a $a$ o a $b$,
  \item cuando el intervalo $(a, b)$ no está acotado. 
 \end{itemize}



  
\end{block}


 \end{frame}
 

\begin{frame}{Clasificación problemas Sturm-Liouville}


\textbf{Ejemplo (problema singular):} Problema contorno para la Ecuación de Bessel en $[0,b]$

\[
 \left\{
        \begin{array}{l}
         \frac{d}{d x}\left[x \frac{d y}{d x}\right]+\lambda x y=0, \quad 0<x<b\\
         \lim _{x \rightarrow 0^{+}} y(x)\hbox{ y }\lim _{x \rightarrow 0^{+}} y^{\prime}(x) \hbox{ existen y son finitos},\\ 
         y(b)=0 .\\         
        \end{array}
 \right.
\]

Este problema surge al estudiar el flujo de calor en un cilindro. En este caso, $p(x)=r(x)=x$, que se anula en $x=0$. 

 \end{frame}

\section{Problemas de contorno} 
 
\subsection{Clasificación condiciones de contorno}
 
\begin{frame}{Problemas contorno ecuaciones lineales ordinarias de segundo orden}

Supongamos el problema de contorno

\[
 \left\{
        \begin{array}{l}
            y^{\prime \prime}+p(x) y^{\prime}+q(x) y=f(x), \quad 0<x<b\\
            a_{11} y(a)+a_{12} y^{\prime}(a)+b_{11} y(b)+b_{12} y^{\prime}(b)=c_{1}\\
            a_{21} y(a)+a_{22} y^{\prime}(a)+b_{21} y(b)+b_{22} y^{\prime}(b)=c_{2}\\ 
         \end{array}
 \right.
\]

Estas son condiciones de contorno lineales. Cuando $c_{1}=c_{2}=0$, decimos que las condiciones de contorno  son \emph{homogéneas}; en caso contrario, son \emph{no homogéneas}.
\end{frame}


\begin{frame}{Clasificación condiciones de contorno}
Ciertas condiciones en la frontera aparecen con frecuencia en las aplicaciones; estas condiciones son

\begin{description}
 \item[Separadas:]<+->
 \[\begin{split}
    a_{1} y(a)+a_{2} y^{\prime}(a)&=c_{1},\\
    b_{1} y(b)+b_{2} y^{\prime}(b)&=c_{2}
   \end{split}
\]
\item[Dirichlet:]<+-> $$y(a)=c_{1},\quad y(b)=c_{2} .$$
\item[Neumann:]<+-> $$y^{\prime}(a)=c_{1},\quad y^{\prime}(b)=c_{2}$$.
\item[Periódicas]<+-> 
$$
\begin{aligned}
&y(-T)=y(T), \quad y^{\prime}(-T)=y^{\prime}(T), \\
&y(0)=y(2 T), \quad y^{\prime}(0)=y^{\prime}(2 T),
\end{aligned}
$$
donde el periodo es $2 T$
\end{description}

\end{frame}

\subsection{Conjunto de soluciones}

\begin{frame}{Conjunto de soluciones}

Hay tres posibilidades para la ecuación homogénea con condiciones homogéneas en la frontera
\[
 \left\{
        \begin{array}{l}
            y^{\prime \prime}+p(x) y^{\prime}+q(x) y=0, \quad 0<x<b\\
            a_{11} y(a)+a_{12} y^{\prime}(a)+b_{11} y(b)+b_{12} y^{\prime}(b)=0\\
            a_{21} y(a)+a_{22} y^{\prime}(a)+b_{21} y(b)+b_{22} y^{\prime}(b)=0\\ 
         \end{array}
 \right.
\]
\onslide<+->
\begin{itemize}


\item<+-> Si $\phi(x)$ es solución no trivial  entonces también lo es $A \phi$ para cualquier  $A\in\rr$. Tenemos una \emph{familia uniparamétrica de soluciones}.


\item<+-> Si $\phi_{1}(x)$ y $\phi_{2}(x)$ son  dos soluciones linealmente independientes, entonces $A_{1} \phi_{1}(x)+A_{2} \phi_{2}(x)$ también es solución para cualquieras  $A_{1}, A_{2}\in\rr$. Tenemos una \emph{familia biparamétrica de soluciones}.


\item<+-> La otra posibilidad es que $\phi(x) \equiv 0$ sea la única solución, en cuyo caso existe una única solución. 

\end{itemize} 
\onslide<+->
\textbf{Conclusión.} Hay tres situaciones: el problema de contorno  tiene una única solución, una familia  uniparamétrica de soluciones, o una familia bi-paramétrica de soluciones. 

\end{frame}


\begin{frame}{Ejemplo 1}
\onslide<+->
Determinar todas las soluciones de
\[
 \left\{
        \begin{array}{l}
                    y^{\prime \prime}+2 y^{\prime}+5 y=0;\\
                    y(0)=2, \quad y(\pi / 4)=1.\\
        \end{array}
 \right.
\]    
\onslide<+->

\emph{Solución.} La ecuación carascterística es:
$$r^{2}+2 r+5=0,$$
que tiene las raíces $r=-1 \pm 2 i$. La solución general es:

$$y(x)=c_{1} e^{-x} \cos 2 x+c_{2} e^{-x} \sen 2 x.$$ 

Determinamos $c_{1}$ y $c_{2}$ usando  las condiciones de contorno 
$$
y(0)=c_{1}=2, \quad y(\pi / 4)=c_{2} e^{-\pi / 4}=1 .
$$
Por consiguiente, $c_{1}=2$ y $c_{2}=e^{\pi / 4}$, y hay solución única
$$
y(x)=2 e^{-x} \cos 2 x+e^{x / 4} e^{-x} \sen 2 x
$$


\end{frame}


\begin{frame}{Ejemplo 2}
\onslide<+->
  Determinar todas las soluciones del problema con valores en la frontera
\[
 \left\{
        \begin{array}{l}
                     y^{\prime \prime}+y=\cos 2 x;\\
                     y^{\prime}(0)=0, \quad y^{\prime}(\pi)=0.
        \end{array}
 \right.
\]    

\onslide<+->
\emph{Solución} Ecuación característica: 
                    $$r^{2}+1=0,$$ 
Solución general para la ecuación homogénea:
$$y_{h}(x)=c_{1} \cos x+c_{2} \sen x.$$
Usamos el método de coeficientes indeterminados. Una solución particular del problema no-homogéneo  tiene la forma 

$$y_{p}(x)=A\cos 2 x+B \sen 2 x.$$ 

\end{frame}


\begin{frame}{Ejemplo 2 (continuación)}
Al sustituir $y_{p}$  y despejar $A$ y $B$, vemos que $A=-1 / 3$ y $B=0$. Por lo tanto, $y_{p}(x)=-(1 / 3) \cos 2 x$. Así, una solución general  es

$$y(x)=c_{1} \cos x+c_{2} \sen x-(1 / 3) \cos 2 x.$$

Sustituimos la solución general en las condiciones de contorno
$$
y^{\prime}(0)=c_{2}=0, \quad y^{\prime}(\pi)=-c_{2}=0 .
$$
Así, $c_{2}=0$ y $c_{1}$ es arbitrario. El problema  tiene una familia uni-paramétrica de soluciones:
$$
y(x)=c_{1} \cos x-(1 / 3) \cos 2 x,\quad c_1\in\rr
$$

\end{frame}


\begin{frame}{Ejemplo 3}
\onslide<+->
Determinar las soluciones de
\[
 \left\{
        \begin{array}{l}
                     y^{''}+4 y=0;\\
                     y(-\pi)=y(\pi), \quad y^{\prime}(-\pi)=y^{\prime}(\pi).
        \end{array}
 \right.
\]    

\onslide<+->
\emph{Solución.} Ecuación característica es 

$$r^{2}+4=0,$$ 
Solución general
$$y(x)=c_{1} \cos 2 x+c_{2} \operatorname{sen} 2 x.$$

Cualquiera sea $c_1$ y $c_2$ se tiene  
$$y(-\pi)=y(\pi),\quad  y^{\prime}(-\pi)=y^{\prime}(\pi).$$ 

Así  hay una familia a bi-parámetrica de soluciones.

\end{frame}


\begin{frame}{Ejemplo 3 (continuación)}

Si, en el ejemplo anterior, reemplazamos la ecuación diferencial (10) por la ecuación no homogénea
$$y^{\prime \prime}+4 y=4 x$$.

Una solución general es
$$y(x)=x+c_{1} \cos 2 x+c_{2} \operatorname{sen} 2 x.$$

Como 
$$y(-\pi)=-\pi+c_{1},\quad y(\pi)=\pi+c_{1},$$ 

no existen soluciones de  que satisfagan $y(-\pi)=y(\pi)$. Así, el problema no homogéneo con valores en la frontera  no tiene soluciones.

\end{frame}

\subsection{Problemas de autovalores}
 
\begin{frame}{Autovalores}


\begin{block}{Objetivo}
Los problemas de Sturm-Liouville con valores en la frontera   son ejemplos de problemas con valores en la frontera en dos puntos que contienen un parámetro $\lambda$. Nuestro objetivo es \emph{determinar para qué valores de $\lambda$   el problema con valores en la frontera}

\begin{empheq}[box=\tcbhighmath]{equation}\label{eq:sl_main1}  
\left\{
        \begin{array}{ll}
                    \frac{d}{d x}\left(p(x) \frac{d y}{d x}\right)+q(x) y+\lambda r(x) y=0, & a<x<b\\
                    a_{1} y(a)+a_{2} y^{\prime}(a)=0,&\\
                     b_{1} y(b)+b_{2} y^{\prime}(b)=0, &
        \end{array}
 \right.
\end{empheq}
 \emph{tiene soluciones no triviales}. Tales problemas se llaman \emph{problemas de valores propios}. Las soluciones no triviales se llaman \emph{funciones propias o autofunciones} y el número correspondiente $\lambda$ es un \emph{valor propio o autovalor}.
\end{block}

\end{frame}


 

\begin{frame}{ Ejemplo 1}
\onslide<+->
La importancia de los problemas de valores propios es que surgen al usar el método de separación de variables para resolver ecuaciones diferenciales parciales.
\onslide<+->

\textbf{Ejemplo }
$$
\left\{\begin{array}{r}
X^{\prime \prime}+\lambda X=0 \\
X(0)=X(L)=0
\end{array}\right.
$$

Se concluye


$$
\lambda=\left(\frac{n \pi}{L}\right)^{2} \quad n=1,2, \ldots
$$
son los autovalores del problema, y las funciones
$$
X_{n}(x)=\operatorname{sen}\left(\frac{n \pi x}{L}\right), \quad n=1,2, \ldots
$$
son las correspondientes autofunciones.
\end{frame}


 

\begin{frame}{ Ejemplo 2}
El problema 
$$
\left\{\begin{array}{l}
X^{\prime \prime}+\lambda X=0\quad 0<x<L \\
X^{\prime}(0)=X^{\prime}(L)=0
\end{array} \right.
$$

tenía los autovalores y autofunciones 
$$
\lambda=\left(\frac{n \pi}{L}\right)^{2} \quad n=0,1,2,3, \ldots \quad \text { y } X(x)=\cos \left(\frac{n \pi x}{L}\right)
$$
respectivamente 



\end{frame}


 

\begin{frame}{Ejemplo 3}
\onslide<+->
 Determinar todos los valores propios reales y funciones propias correspondientes para
 
 $$
\left\{\begin{array}{l}
            y^{\prime \prime}+\lambda y=0;\\
            y(0)=0, \quad 3 y(\pi)-y^{\prime}(\pi)=0.
\end{array} \right.
$$
\onslide<+->

\textbf{Solución.} Ecuación característica 

$$r^{2}+\lambda=0.$$

Hay tres casos

\textbf{Caso 1. $\lambda=-\mu^{2}<\mathbf{0}$.} Raíces:

$$r=\pm \mu,$$

Solución general:

$$y(x)=c_{1} e^{\mu x}+c_{2} e^{-\mu x}.$$ 
\end{frame}


 

\begin{frame}{Ejemplo 3 (continuación)}
Equivalentemente 

$$y(x)=C_{1} \cosh \mu x+C_{2} \senh \mu x.$$

Al sustituir en las condiciones en la frontera:
$$
C_{1}=0, \quad 3\left(C_{1} \cosh \mu \pi+C_{2} \operatorname{senh} \mu \pi\right)-\left(\mu C_{1} \operatorname{senh} \mu \pi+\mu C_{2} \cosh \mu \pi\right)=0
$$

Para $C_{1}=0$, la última ecuación se convierte en
$$
C_{2}(3 \operatorname{senh} \mu \pi-\mu \cosh \mu \pi)=0 .
$$
Para obtener una solución no trivial, debemos tener $C_{2} \neq 0$, de modo que:

$$3 \senh \mu \pi-\mu\cosh \mu \pi=0;$$
es decir, $\mu$ debe satisfacer
 $$\tanh \mu \pi=\frac{1}{3} \mu.$$
 
 \end{frame}


 

\begin{frame}{Ejemplo 3 (continuación)}
 

En el plano $\mu y$, la recta $y=\mu / 3$ corta a la curva $y=\tanh \mu \pi$ sólo una vez para $\mu>0$ (véase la figura). Por lo tanto, sólo existe una solución positiva de, que denotaremos $\mu_{0}$.


\includegraphics[scale=.8]{/home/fernando/fer/Docencia/posgrado/Derivadas Parciales Maestría/2022/unidad2-7/tanh.png}

En resumen, el problema con valores en la frontera (20)-(21) tiene un valor propio negativo $\lambda_{0}=-\mu_{0}^{2}$, tal que $\tan \mu_{0} \pi=\mu_{0} / 3$, y las funciones propias correspondientes son $y_{0}(x)=c_{0} \operatorname{senh} \mu_{0} x$, con $c_{0} \neq 0$ arbitrario.
\end{frame}


\begin{frame}{Ejemplo 3 (continuación)}
 

\textbf{Conclusión.} En el caso $\lambda<0$ el problema con valores en la frontera tiene un valor propio negativo 
$$\lambda_{0}=-\mu_{0}^{2},$$
tal que 
$$\tan \mu_{0} \pi=\mu_{0} / 3,$$
y las funciones propias correspondientes son 

$$y_{0}(x)=c_{0} \operatorname{senh} \mu_{0} x,\quad c_{0} \in\rr.$$ 

 \end{frame}


 

\begin{frame}{Ejemplo 3 (continuación)}
\textbf{Caso 2. $\lambda=0$.} Cero es una raíz doble de la ecuación auxiliar, la solución general es:

$$y(x)=c_{1} x+c_{2}.$$ 

Al sustituir en las condiciones de contorno
$$
c_{2}=0, \quad 3 \pi c_{1}+3 c_{2}-c_{1}=0
$$
que tiene la solución $c_{1}=c_{2}=0$. 

No tenemos funciones propias.
 \end{frame}


 

\begin{frame}{Ejemplo 3 (continuación)}
\textbf{Caso 3. $\lambda=\mu^{2}>0$  para $\mu>0$.} Raíces:

$$r=\pm \mu i,$$
Solución general: 

$$y(x)=c_{1} \cos \mu x+c_{2}\sen \mu x .$$

Sustituyenco en condiciones de contorno
$$
c_{1}=0, \quad 3\left(c_{1} \cos \mu \pi+c_{2} \operatorname{sen} \mu \pi\right)-\left(-\mu c_{1} \operatorname{sen} \mu \pi+\mu c_{2} \cos \mu \pi\right)=0 .
$$
$\mathrm{Al}$ hacer $c_{1}=0$ en la última ecuación, obtenemos
$$
c_{2}(3 \operatorname{sen} \mu \pi-\mu \cos \mu \pi)=0 .
$$
Para que existan soluciones no triviales, $\mu$ debe satisfacer 

$$\tan \mu \pi=\frac{1}{3} \mu.$$

 \end{frame}


 

\begin{frame}{Ejemplo 3 (continuación)}
\includegraphics[scale=.6]{/home/fernando/fer/Docencia/posgrado/Derivadas Parciales Maestría/2022/unidad2-7/tanh2.png}

Hay una infinidad de soluciones $\mu_{1}<\mu_{2}<\ldots $ por ende de autovalores $\lambda_{n}=\mu_{n}^{2}, n=1,2,3, \ldots$ con autofunciones correspondientes  

$$y_{n}(x)=c_{2}\sen \mu_{n} x, \hbox{ con }c_{2} \neq 0,$$  



\end{frame}


\section{Problemas regulares de Sturm-Liouville}



 

\begin{frame}{Problemas regulares de Sturm-Liouville}
\onslide<+->
Retornemos al problema general de hallar los valores propios y funciones propias de:
\begin{empheq}[box=\tcbhighmath]{equation}\label{eq:sl_main2}  
\left\{
        \begin{array}{ll}
                    \frac{d}{d x}\left(p(x) \frac{d y}{d x}\right)+q(x) y+\lambda r(x) y=0, & a<x<b\\
                    a_{1} y(a)+a_{2} y^{\prime}(a)=0,&\\
                     b_{1} y(b)+b_{2} y^{\prime}(b)=0, &
        \end{array}
 \right.
\end{empheq}
donde
\begin{itemize}
 \item<+-> $p(x), p^{\prime}(x), q(x)$ y $r(x)$ son funciones continuas en $[a, b]$
 \item<+-> $p(x)>0$ y $r(x)>0$ en $[a, b]$.
 \item<+-> Se excluye el caso en que $a_{1}=a_{2}=0$ o $b_{1}=b_{2}=0$.
\end{itemize}   
\end{frame}


\subsection{Llevando a la forma de Sturm-Liouville}



 

\begin{frame}{Llevando a la forma de Sturm-Liouville}
\onslide<+->
\textbf{Disgresión:} cualquier ecuación de la forma 
\begin{equation}\label{eq:20gral}
    A_{2}(x) y^{\prime \prime}(x)+A_{1}(x) y^{\prime}(x)+A_{0}(x) y(x)+\lambda \rho(x) y(x)=0
\end{equation}
se puede convertir  en una ecuación del tipo que \eqref{eq:sl_main2}.


\onslide<+->
Debemos determinar $p$ de modo que 

$$\left(p y^{\prime}\right)^{\prime}=p y^{\prime \prime}+p^{\prime} y^{\prime}=A_{2} y^{\prime \prime}+A_{1}y^{\prime}.$$

Es decir 

$$p=A_{2},\quad p^{\prime}=A_{1}.$$
Pero en general, $A_{2}^{\prime} \neq A_{1}$, de modo que este método directo no siempre es aplicable.
\end{frame}

 



 

\begin{frame}{Llevando a la forma de Sturm-Liouville}

\tikz[baseline]{
      \node[fill=frametit, square,anchor=base] (t0)
           {Idea!}} Buscar
 \emph{factor integrante} $\mu(x)$ tal que al multiplicar \eqref{eq:20gral} por $\mu$,  obtenemos coeficientes tales que  $\left(\mu A_{2}\right)^{\prime}=\mu A_{1}$. 
 
 
 Lamemos $p=\mu A_{2}$,  queremos 
 
 $$p^{\prime}=\mu A_{1}=p A_{1} / A_{2}.$$
 
 Es una ecuación en variables separables. Resolviendo
$$
p(x)=C e^{\int A_{1}(x) / A_{2}( x) d x},\quad C\in\rr
$$

Tomamos $C=1$ 
$$\mu(x)=p / A_{2}=\left[1 / A_{2}(x)\right] e^{\int A_{1}(x) / A_{2}(x) d x}.$$

Al multiplicar \eqref{eq:20gral} por  $\mu$ se tiene

$$\left(p y^{\prime}\right)^{\prime}+q y+\lambda r y=0,$$

donde $p=\mu A_{2}, q=\mu A_{0} \mathrm{y} r=\mu \rho$. Necesitamos  que $A_{2}(x)\neq 0$

\end{frame}

 



 

\begin{frame}{Ejemplo}
 \onslide<+->
 Convertir la siguiente ecuación a la forma de una ecuación de Sturm-Liouville:
$$
 3 x^{2} y''(x)+4 x y^{\prime}(x)+6 y(x)+\lambda y(x)=0, \quad x>0.
$$

\onslide<+->
\textbf{Solución.}  $A_{2}(x)=3 x^{2}$ y $A_{1}(x)=4 x$.
$$
\begin{aligned}
\mu(x) &=\frac{1}{3 x^{2}} e^{\int A_1(x) / A_2(x) d x}=\frac{1}{3 x^{2}} e^{\int(4 x)/3 x^{2} d x} \\
&=\frac{1}{3 x^{2}} e^{(4 / 3) \int x^{-1} d x}=\frac{1}{3 x^{2}} e^{(4 / 3) \ln x}=\frac{x^{4 / 3}}{3 x^{2}}=\frac{1}{3 x^{2 / 3}} .
\end{aligned}
$$
Al multiplicar la ecuación por $\mu(x)=1 /\left(3 x^{2 / 3}\right)$, obtenemos

$$\left(x^{4 / 3} y^{\prime}(x)\right)^{\prime}+2 x^{-2 / 3} y(x)+\lambda\left(3 x^{2 / 3}\right)^{-1} y(x)=0.$$

\end{frame}

 


\subsection{Identidades Lagrange-Green}
 

\begin{frame}{Identidad de Lagrange}\onslide<+->
Definimos

$$\tikz[baseline]{
      \node[fill=frametit, square,anchor=base] (t0)
           {$ L\left[y](x):=\left(p(x) y^{\prime}(x)\right)^{\prime}+q(x) y(x)\right.$ }} $$
      
 la ecuación se escribe 
$$\tikz[baseline]{
      \node[fill=frametit, square,anchor=base] (t0)
           {$L[y](x)+\lambda r(x) y(x)=0 .$ }} 
$$
\onslide<+->

\begin{block}{Teorema [Identidad de lagrange]} Suponmgamos que $p$ y $q$ son funciones continuas en $[a,b]$ con \emph{valores en $\mathbb{R}$}.   Sean $u$ y $v$ funciones con segundas derivadas continuas en el intervalo $[a, b]$ con \emph{valores en $\mathbb{C}$}. Entonces,

$$\tikz[baseline]{
      \node[fill=frametit, square,anchor=base] (t0)
           {$u L[v]-L[u]v=\frac{d}{d x}(p W[u, v]),$
    }} $$
donde el \emph{wronskiano} de $u$ y $v$ se define
$W[u, v]=u v^{\prime}-v u^{\prime}$  
\end{block}
\end{frame}

 



\begin{frame}{Identidad de Lagrange (Demostración)}

Usamos la regla del producto y sumamos y restamos $p u^{\prime} v^{\prime}$ para tener
$$
\begin{aligned}
u L[v]-v L[u] &=u\left[\left(p v^{\prime}\right)^{\prime}+q v\right]-v\left[\left(p u^{\prime}\right)^{\prime}+q u\right] \\
&=u\left(p v^{\prime}\right)^{\prime}+q u v-v\left(p u^{\prime}\right)^{\prime}-q u v \\
&=u\left(p v^{\prime}\right)^{\prime}+u^{\prime}\left(p v^{\prime}\right)-v^{\prime}\left(p u^{\prime}\right)-v\left(p u^{\prime}\right)^{\prime} \\
&=\left[u\left(p v^{\prime}\right)\right]^{\prime}-\left[v\left(p u^{\prime}\right)\right]^{\prime} \\
&=\left[ p\left(u v^{\prime}-v u^{\prime}\right)\right]^{\prime} \\
&=\frac{d}{d x}[p W[u, v]]
\end{aligned}
$$

\end{frame}

 



\begin{frame}{Fórmula de Green}


\begin{block}{Corolario [Fórmula de Green]}
Bajo las hipótesis del Teorema anterior 
 
 \begin{empheq}[box=\tcbhighmath]{equation}\label{eq:for_green_1}\int_{a}^{b}(u L[v]-v L[u])(x) d x=\left.(p W[u, v])(x)\right|_{a} ^{b}.
\end{empheq}
 
Si además u y $v$ satisfacen las condiciones en la frontera de \eqref{eq:sl_main2}, la fórmula de Green se simplifica a


\begin{empheq}[box=\tcbhighmath]{equation}\label{eq:for_green_2}\int_{a}^{b}(u L[v]-v L[u])(x) d x=0.
\end{empheq}
\end{block}
\end{frame}

 



\begin{frame}{Fórmula de Green (Demostración)}

La primera fórmula surge de integrar la identidad de Lagrange.

Para la segunda notar que si $a_{2} \neq 0$ y $b_{2} \neq 0$, entonces
$$
\begin{array}{ll}
u^{\prime}(a)=-\left(a_{1} / a_{2}\right) u(a), & u^{\prime}(b)=-\left(b_{1} / b_{2}\right) u(b) \\
v^{\prime}(a)=-\left(a_{1} / a_{2}\right) v(a), & v^{\prime}(b)=-\left(b_{1} / b_{2}\right) v(b)
\end{array}
$$
Al sustituir estos valores en el lado derecho de la primera fórmula
$$
\begin{aligned}
p(b) W & {[u, v](b)-p(a) W[u, v](a) } \\
=& p(b)\left[u(b) v^{\prime}(b)-u^{\prime}(b) v(b)\right]-p(a)\left[u(a) v^{\prime}(a)-u^{\prime}(a) v(a)\right] \\
=& p(b)\left[u(b)\left(-b_{1} / b_{2}\right) v(b)-\left(-b_{1} / b_{2}\right) u(b) v(b)\right] \\
& \quad-p(a)\left[u(a)\left(-a_{1} / a_{2}\right) v(a)-\left(-a_{1} / a_{2}\right) u(a) v(a)\right] \\
=& 0
\end{aligned}
$$

\end{frame}


 



\begin{frame}{Fórmula de Green (Demostración)}
 Cuando $a_{2}=0$ (de modo que $a_{1} \neq 0$ ), las condiciones de contorno implican que $u(a)$ y $v(a)$ se anulan. Por lo tanto, 
 $$\left[u(a) v^{\prime}(a)-u^{\prime}(a) v(a)\right]=0.$$
 De manera análoga, cuando $b_{2}=0$, obtenemos 
 $$\left[u(b) v^{\prime}(b)-u^{\prime}(b) v(b)\right]=0.$$\qed
\end{frame}




\section{Nociones básicas análisis funcional}
\subsection{Espacio $L^2_r([a,b])$}

\begin{frame}{Espacio $L^2_r([a,b])$}

\begin{block}{Definición [Espacio $L^2_r([a,b])$]}
Dado un intervalo $\left[a, b\right] \subset \mathbb{R}$ y $w:[a,b]\to\mathbb{R}$  positiva en $(a,b)$, definimos
$$
L^2_r([a, b])=\left\{f:[a,b]\to\mathbb{C} \bigg| \int_a^b |f(x) |^2r(x)d x<\infty\right\}
$$
Si $f\in L^2_r([a,b])$:

$$\|f\|_{L^2_r}= \left\{\int_a^b |f(x) |^2r(x)d x\right\}^{\frac12}.$$

\emph{Asumiremos las funciones continuas a trozos}

\end{block}
\end{frame}




\subsection{producto interno}

\begin{frame}{Producto interno}

\begin{block}{Producto interno}
Para $f,g\in L^2_r([a,b])$ se define

\begin{empheq}[box=\tcbhighmath]{equation}\label{eq:pro_int}
 \langle f\mid g\rangle_r=\int_{a}^{b} \overline{f(x)} g(x)r(x) d x.
\end{empheq}



\end{block}

\emph{Convención de notación} Cuando $r\equiv 1$ vamos a escribir por simplicidad $L^2_1([a,b])=L^2([a,b])$ y $\langle\cdot\mid \cdot\rangle_1= \langle\cdot\mid \cdot\rangle$. 

\end{frame}




\begin{frame}{Espacio $L^2_r([a,b])$.}


\begin{block}{Teorema} $L^2_r([a, b])$ es un espacio vectorial. 
\end{block}

\textbf{Demostración} si $f, g \in L^2_r([a, b])$ y  $\alpha \in \mathbb{R}$.

$$
\int_a^b\left|\alpha f\right|^2 r(x)d x=\alpha^2 \int_a^b \left|f(x)\right|^2 r(x)d x<\infty.
$$
Luego $\alpha f \in L^2_r([a, b])$. 
\end{frame}




\begin{frame}{Espacio $L^2_r([a,b])$.}


Usamos 
\begin{equation}\label{eq:youg}
 a b \leq \frac{a^2}{2}+\frac{b^2}{2},\quad  a \geq 0, b \geq 0.
\end{equation}
Luego

\[
\begin{split}
\int_a^b |f+ g|^2r(x)dx &\leq  \int_a^b|f|^2r(x)dx+2\int_a^b|f||g|r(x)dx+\int_a^b|g|^2r(x)dx \\
&\leq 2\int_a^b|f|^2r(x)dx+2\int_a^b|g|^2r(x)dx<\infty.
\end{split}
\]
Luego $f+g\in L^2_r([a,b])$.\qed


\end{frame}







\begin{frame}{Propiedades producto Interno}


\begin{block}{Propiedades Producto Interno} $\langle\cdot\mid \cdot\rangle_r:L^2_r([a,b])\times L^2_r([a,b])\to\rr$ satisface para $f,g,h\in L^2([a,b])$ y $\alpha\in\mathbb{C}$:  
\begin{description}
 \item[Multilinealidad]
 
 \begin{itemize}
                         \item $\langle f+g\mid h\rangle_r=\langle f \mid h\rangle_r +\langle g\mid  h\rangle_r$, $ \langle h \mid f+g\rangle_r=\langle h \mid f\rangle_r+\langle h\mid g\rangle_r$
                         \item $\langle \alpha f \mid g\rangle_r =\overline{\alpha}\langle f \mid g\rangle_r $, $\langle f\mid\alpha g\rangle_r =\alpha\langle f\mid g\rangle_r $
                        \end{itemize}
\item[Simetría]  $\langle f\mid g\rangle_r =\overline{\langle g\mid f\rangle_r }$
\item[No degeneración] $\langle f\mid f\rangle_r \geq 0$ y $\langle f\mid f\rangle_r  =0\Leftrightarrow f=0$.
\item[Desigualdad de Cauchy-Schwartz] 

\begin{empheq}[box=\tcbhighmath]{equation}\label{eq:cauchy-scha}
 |\langle f\mid g \rangle_r|\leq \|f\|_{L^2_r} \| g\|_{L^2_r}.
\end{empheq}
 
                        
\end{description}

 
\end{block}


\end{frame}

\begin{frame}{Propiedades producto Interno}


Sólo demostraremos la desigualdad de Cauchy-Schwartz. Las otras propiedades son sencillas.


Asumimos $f, g \neq 0$. Entonces 
$$
\frac{|f|}{\|f\|_{L^2_r}} \frac{|g|}{\|g\|_{L^2_r}} \leqslant \frac{1}{2\|f\|_{L^2_r}^2}|f|^2+\frac{1}{2 \| g\|_{L^2_r}^2}|g|^2
$$
Multiplicando por $r(x)$ e integrando
$$
\int_a^b \frac{|f|}{\|f\|_{L^2_r}} \frac{|g|}{\|g\|_{L^2_r}} r(x)d x \leq \frac{1}{2\|f\|_{L^2_r}^2} \int_a^b|f|^2r(x) d x+\frac{1}{2 \| g\|_{L^2_r}^2} \int_a^b \left| g\right|^2r(x)dx=1
$$
De aca sale \qed
\end{frame}





\begin{frame}{Propiedades de la norma}
\begin{block}{Propiedades norma} S $f,g\in L^2([a,b],r)$ y $\alpha\in\mathbb{C}$:  
\begin{description}
 

 \item[No degeneración] $\|f\|_{L^2_r}\geq 0$ y $\|f\|_{L^2_r}=0\Leftrightarrow f=0$.
\item[Homogeneidad] $\|\alpha f\|_{L^2_r}=|\alpha|\|f\|_{L^2_r}$.
\item[Desigualdad Triangular o Minkowski] 
\begin{empheq}[box=\tcbhighmath]{equation}\label{eq:minkowski}
 \|f+ g\|_{L^2_r}\leq \|f\|_{L^2_r} + \| g\|_{L^2_r}.
\end{empheq}
 \end{description}
\end{block}



\end{frame}

 


\begin{frame}{Propiedades de la norma}
Sólo demostraremos la desigualdad de Minkowski

$$
\begin{aligned}
\|f+g \|_{L^2_r}^2 
&= \int_a^b\overline{(f+g)} (f+g)r(x)d x
=\int_a^b \left(|f|^2+2\operatorname{Re}( f \overline{g})+|g|^2\right) r(x)d x . \\
& \leq \int_a^b |f|^2 r(x)d x+2\int_a^b|f\overline{g}|r(x)dx+\int_a^b |g|^2 r(x)d x \\
&=\|f\|_{L^2_r}^2+2\|f\|_{L^2_r}\|g\|_{L^2_r}+\|g\|_{L^2_r}^2=(\|f\|_{L^2_r}+\|g\|_{L^2_r})^2
\end{aligned}
$$

\end{frame}

\subsection{Operadores autoadjuntos}

\begin{frame}{Autoadjunción} \onslide<+->
Si $v$ satisface las condiciones de contorno \eqref{eq:sl_main2}, $\overline{v}$ también lo hace.  Luego si aplicamos  \eqref{eq:for_green_2}  con  $\overline{v}$ en lugar de $v$ y usamos que $L[\overline{v}]=\overline{L[v]}$  vemos que 

\begin{equation}\label{eq:autoadjunto}
 \tikz[baseline]{
      \node[fill=frametit] (t0)
           {$\langle u\mid L[v]\rangle = \langle L[u]\mid v\rangle.$
    }}
\end{equation}

Recordar que aquí $r\equiv 1$.
\onslide<+->

\begin{block}{Definición [Operadores autoadjuntos]} Un operador diferencial lineal $L$ que satisface \eqref{eq:autoadjunto} para todas las funciones $u$ y $v$ en un espacio vectorial $V$ se llama un \emph{operador autoadjunto. } o \emph{hermitiano} sobre $V$.
\end{block}

\onslide<+->

\textbf{Observación.} Hemos mostrado que si $L[y]=(py')'+qy$  y  $V$ es el conjunto de funciones que tienen segundas derivadas continuas en $[a, b]$ y satisfacen las condiciones en la frontera en \eqref{eq:sl_main2} , entonces $L$ es un operador autoadjunto en $V$.

\end{frame}

\section{Propiedades operadopres autoadjuntos}
\subsection{Autoadjunción $\Rightarrow \lambda\in\rr$}  
\begin{frame}{Autoadjunción $\Rightarrow \lambda\in\rr$}
\onslide<+->
\begin{block}{Teorema} Los valores propios de un problema regular de Sturm-Liouville \eqref{eq:sl_main2}  son reales y tienen funciones propias con valores reales.
 
\end{block}

\onslide<+->
\textbf{Demostración.} Supongamos que $\lambda\in \mathbb{C}$ es un valor propio, con función propia $\phi(x)$ a valores complejos. Es decir,
$$L[\phi](x)+\lambda r(x) \phi(x)=0.$$

y $\phi\not\equiv 0$ satisface las condiciones en la frontera en  \eqref{eq:sl_main2}. Como $p, q$ y $r$ asumen valores reales, obtenemos
$$\overline{L[\phi](x)+\lambda r(x) \phi(x)}=L[\bar{\phi}](x)+\bar{\lambda} r(x) \bar{\phi}(x)=0$$
Como $a_{1}, a_{2}, b_{1}$ y $b_{2}$ son reales $\bar{\phi}$ también satisface las condiciones en la frontera en \eqref{eq:sl_main2}. Por lo tanto, $\bar{\lambda}$ es un valor propio con función propia $\bar{\phi}$

\end{frame}


\begin{frame}{Autoadjunción $\Rightarrow \lambda\in\rr$}

Por otro lado
$$\int_{a}^{b} L[\phi](x) \bar{\phi}(x) d x=-\lambda \int_{a}^{b} r(x) \phi(x) \bar{\phi}(x) d x=-\lambda \int_{a}^{b} r(x)|\phi(x)|^{2} d x.$$


Además por la segunda identidad de Green y el hecho de que $\bar{\lambda}$ sea un valor propio con función propia $\bar{\phi}$, vemos que


$$
\int_{a}^{b} L[\phi](x) \bar{\phi}(x) d x=\int_{a}^{b} \phi(x) L[\bar{\phi}](x) d x=-\bar{\lambda} \int_{a}^{b} r(x)|\phi(x)|^{2} d x
$$
Pero los lados izquierdos de las ecuaciones anteriores son los mismos, de modo que 
$$
-\lambda \int_{a}^{b} r(x)|\phi(x)|^{2} d x=-\bar{\lambda} \int_{a}^{b} r(x)|\phi(x)|^{2} d x
$$

\end{frame}


\begin{frame}{Autoadjunción $\Rightarrow \lambda\in\rr$}
Como $r(x)>0$ y $\phi\neq 0$, entonces 
$$\int_{a}^{b} r(x)|\phi(x)|^{2} d x>0.$$ 

Obtenemos $\lambda=\bar{\lambda}$, vale decir $\lambda\in\rr$. 

Tomando partes real e imaginaria de  $\phi$ obtenemos  funciones propias con valores reales correspondientes a $\lambda$.\qed 
\end{frame}

\subsection{Multiplicidad de autovalores}

\begin{frame}{Multiplicidad de autovalores}

\onslide<+->
\begin{block}{Definición [Autovalores simples]} Si todas las funciones propias asociadas a un valor propio particular son sólo múltiplos escalares entre sí, entonces el valor propio se llama \emph{simple}.
\end{block}

\onslide<+->
\begin{block}{Teorema [Autovalores simples]}
Todos los valores propios del problema regular de Sturm-Liouville con valores en la frontera \eqref{eq:sl_main2} son simples.
\end{block}
\end{frame}


\begin{frame}{Multiplicidad de autovalores}

\textbf{Demostración.} Si $\phi(x)$ y $\psi(x)$ son autofunciones  correspondientes a $\lambda$: 
$$
a_{1} \phi(a)+a_{2} \phi^{\prime}(a)=0, \quad a_{1} \psi(a)+a_{2} \psi^{\prime}(a)=0
$$
Supongamos que $a_{2} \neq 0$. ( si $a_{2}=0$ y $a_{1} \neq 0$ queda de \textbf{ejercicio}.) Despejando
$$
\phi^{\prime}(a)=\left(-a_{1} / a_{2}\right) \phi(a), \quad \psi^{\prime}(a)=\left(-a_{1} / a_{2}\right) \psi(a)
$$
Calculemos el wronskiano de $\phi$ y $\psi$ en $x=a$
$$
\begin{aligned}
W[\phi, \psi](a) &=\phi(a) \psi^{\prime}(a)-\phi^{\prime}(a) \psi(a) \\
&=\phi(a)\left(-a_{1} / a_{2}\right) \psi(a)-\left(-a_{1} / a_{2}\right) \phi(a) \psi(a)=0
\end{aligned}
$$
Si el wronskiano de dos soluciones de una ecuación lineal homogénea de segundo orden se anula en un punto, entonces las dos soluciones son linealmente dependientes. Así, $\lambda$ es un valor propio simple.\qed

\end{frame}

\subsection{Ortogonalidad}

\begin{frame}{Ortogonalidad}
\begin{block}{Teorema [Ortogonalidad]} Las funciones propias que corresponden a valores propios distintos
de un problema regular de Sturm-Liouville con valores en la frontera \eqref{eq:sl_main2} son \emph{ortogonales} con respecto de la función de ponderación $r(x)$ en $[a, b]$. Vale decir si $\phi$ y $\psi$ son autofunciones asociadas a los autovalores $\lambda$ y $\mu$ con $\lambda\neq \mu$ entonces

\begin{empheq}[box=\tcbhighmath]{equation}\label{eq:ortogo}
 \langle \phi\mid\psi\rangle_r=\int_{a}^{b} \phi(x) \psi(x) r(x) d x=0.
\end{empheq} 
\end{block}

 \end{frame}


\begin{frame}{Ortogonalidad}


\textbf{Demostración.} Tenemos

$$L[\phi]=-\lambda r \phi\quad\hbox{ y }\quad L[\psi]=-\mu r \psi.$$

Podemos suponer que $\phi$ y $\psi$ tienen valores reales. Por la fórmula de Green
$$
\begin{aligned}
0 &=\int_{a}^{b}(\phi L[\psi]-\psi L[\phi])(x) d x \\
&=\int_{a}^{b}(-\phi \mu r \psi+\psi \lambda r \phi)(x) d x \\
&=(\lambda-\mu) \int_{a}^{b} \phi(x) \psi(x) r(x) d x
\end{aligned}
$$
Como $\lambda \neq \mu$, tenemos
$$
\int_{a}^{b} \phi(x) \psi(x) r(x) d x=0
$$
\qed
 \end{frame}


\begin{frame}{Ortogonalidad, Ejemplo}

\textbf{Ejercicio} Demostrar que el problema
$$y^{\prime \prime}+\lambda y=0 ; \quad y(0)=y(\pi)=0$$

tiene autovalores  $\lambda_{n}=n^{2}, n=1,2$, $3, \ldots$, con funciones propias correspondientes $\phi_{n}(x)=c_{n}$ sen $n x$. Verificar la ortogonalidad de manera directa.

 \end{frame}
 
 
 \subsection{Desarrollo en serie de autofunciones}
 
 \begin{frame}{Infinitud de autovalores}
 
\onslide<+->
 \begin{block}{Teorema} Los valores propios del problema regular de Sturm-Liouville con valores en la frontera \eqref{eq:sl_main2} forman una sucesión numerable y creciente
$$\lambda_{1}<\lambda_{2}<\lambda_{3}<\cdots$$
con:
$$\lim _{n \rightarrow \infty} \lambda_{n}=+\infty.$$
\end{block}


\onslide<+->
\textbf{Demostración}. Será omitida, ver por ejemplo Garrett Birkhoff y Gian-Carlo Rota. Ordinary Differential Equations

\end{frame}

 \begin{frame}{Desarrollos en serie}
 Si $\lambda_{1}<\lambda_{2}<\lambda_{3}<\cdots$ son valores propios del problema \eqref{eq:sl_main2} con funciones propias  $\left\{\phi_{n}(x)\right\}_{n=1}^{\infty}$,  ortogonales respecto  a $r(x)$ en $[a, b]$. Como
 $$\frac{\phi_n}{\|\phi_n\|_{L^2_r}},$$
 también es autofunción, se puede asumir que $\{\phi_n\}$ son \emph{ortonormales}
 
 

 $$\int_{a}^{b} \phi_{n}(x) \phi_{m}(x) r(x) d x= \begin{cases}0, & m \neq n, \\ 1, & m=n,\end{cases}.$$ 

Ahora vamos a asociar a una función  $f$ un denominado \emph{desarrollo ortogonal}. 

\begin{empheq}[box=\tcbhighmath]{equation}\label{eq:des_ortogo}
f(x) \sim \sum_{n=1}^{\infty} c_{n} \phi_{n}(x),\quad\text{donde } c_{n}=\int_{a}^{b} f(x) \phi_{n}(x) r(x) d x.
\end{empheq}




\end{frame}

 \begin{frame}{Ortonormalización}

 
\textbf{Ejercicio}  Considerar el problema de Sturm-Liouville con valores en la 
$$y^{\prime \prime}+\lambda y=0 ; \quad y(0)=0, \quad y^{\prime}(\pi)=0.$$

\begin{itemize}
 \item<+-> $\lambda_{n}=$ $(2 n-1)^{2} / 4, n=1,2,3, \ldots$ valores propios.
 \item<+-> $
\phi_{n}(x)=a_{n} \operatorname{sen}\left(\frac{2 n-1}{2}\right) x
$ funciones propias ortogonles respecto $r(x) \equiv 1$ en $[0, \pi]$.
\item<+-> Si $a_{n}=\sqrt{2 / \pi}$ el sistema es ortonormal.
\end{itemize}

\end{frame}

 \begin{frame}{Desarrollos en serie}
 \onslide<+->
 \begin{block}{Teorema}  Sea $\left\{\phi_{n}\right\}_{n=1}^{\infty}$ un sistema ortonormal de funciones propias para el problema regular de Sturm-Liouville con valores en la frontera \eqref{eq:sl_main2}, entonces $\left\{\phi_{n}\right\}_{n=1}^{\infty}$ es completo en el conjunto de funciones continuas de cuadrado integrable en $[a,b]$, es decir si $f$ es continua:
  $$f(x)=\sum_{n=1}^{\infty} c_{n} \phi_{n}(x), \quad a \leq x \leq b,$$
  donde la convergencia de la serie es en media. Si además $f^{\prime}$ continua por partes  en $[a, b]$ y satisface las condiciones en la frontera la serie converge uniformemente en $[a, b]$.
 \end{block}
 
 \onslide<+->
 \textbf{Demostración.}  Será omitida, ver por ejemplo Garrett Birkhoff y Gian-Carlo Rota. Ordinary Differential Equations
 \end{frame}

 \begin{frame}{Desarrollos en serie}
Expresar
$$\small
f(x)= \begin{cases}2 x / \pi, & 0 \leq x \leq \pi / 2 \\ 1, & \pi / 2 \leq x \leq \pi\end{cases}
$$
mediante  el sistema:
$$\small
\phi_{n}(x)=\sqrt{\frac{2}{\pi}}\operatorname{sen}\left(\frac{2 n-1}{2}\right) x. $$ 
$$\small
\begin{aligned}
c_{n}=& \int_{0}^{\pi} f(x) \sqrt{2 / \pi} \operatorname{sen}\left(\frac{2 n-1}{2} x\right) d x \\
=&\left(\frac{2}{\pi}\right)^{3 / 2} \int_{0}^{\pi / 2} x \operatorname{sen}\left(\frac{2 n-1}{2} x\right) d x+\left(\frac{2}{\pi}\right)^{1 / 2} \int_{\pi / 2}^{\pi} \operatorname{sen}\left(\frac{2 n-1}{2} x\right) d x \\
=&\left.\left(\frac{2}{\pi}\right)^{3 / 2}\left[\frac{-2 x}{2 n-1} \cos \left(\frac{2 n-1}{2} x\right)+\frac{4}{(2 n-1)^{2}} \operatorname{sen}\left(\frac{2 n-1}{2} x\right)\right]\right|_{0} ^{\pi / 2} \\
&-\left.\left(\frac{2}{\pi}\right)^{1 / 2}\left[\frac{2}{2 n-1} \cos \left(\frac{2 n-1}{2} x\right)\right]\right|_{\pi / 2} ^{\pi},
\end{aligned}
$$
 \end{frame}

 \begin{frame}{Desarrollos en serie}
 
 Entonces 
$$c_{n}=\frac{2^{7 / 2}}{\pi^{3 / 2}(2 n-1)^{2}} \operatorname{sen}\left(\frac{n \pi}{2}-\frac{\pi}{4}\right),\quad n=1,2,3, \ldots .$$
Por lo tanto,
$$\small
\begin{aligned}
f(x) &=\sum_{n=1}^{\infty} c_{n} \sqrt{2} / \pi \operatorname{sen}\left(\frac{2 n-1}{2} x\right) \\
&=\frac{2^{7 / 2}}{\pi^{2}} \operatorname{sen}(x / 2)+\frac{2^{7 / 2}}{9 \pi^{2}} \operatorname{sen}(3 x / 2)-\frac{2^{7 / 2}}{25 \pi^{2}} \operatorname{sen}(5 x / 2)-\frac{2^{7 / 2}}{49 \pi^{2}} \operatorname{sen}(7 x / 2)+\cdots,
\end{aligned}
$$
 Como $f(0)=0$ y $f^{\prime}(\pi)=0, f$ satisface las condiciones en la frontera. Además, $f$ es continua y $f^{\prime}$ es continua por partes en $[0, \pi]$. Por lo tanto, la serie en converge uniformemente a $f$ en $[0, \pi]$.


 \end{frame}

 \begin{frame}{Otro resultado de convergencia}

\begin{block}{Teorema} Sea $\left\{\phi_{n}\right\}_{n=1}^{\infty}$ una sucesión ortonormal de funciones propias para el problema regular de Sturm-Liouville con valores en la frontera \eqref{eq:sl_main2}. Sean $f$ y $f^{\prime}$ continuas por partes en $[a, b]$. Entonces, para cualquier $x$ en $(a, b)$,
\begin{empheq}[box=\tcbhighmath]{equation}\label{eq:conv_dis}
\sum_{n=1}^{\infty} c_{n} \phi_{n}(x)=\frac{1}{2}\left[f\left(x^{+}\right)+f\left(x^{-}\right)\right],
\end{empheq}
donde las $c_{n}$ están dadas por la fórmula \eqref{eq:des_ortogo}. 
\end{block}
 \end{frame}

\subsection{Problemas periodicos de Sturm-Liouville}

\begin{frame}{Problemas periodicos de Sturm-Liouville}
 
\begin{empheq}[box=\tcbhighmath]{equation}\label{eq:pro_perio}  
\left\{
        \begin{array}{ll}
                    \frac{d}{d x}\left(p(x) \frac{d y}{d x}\right)+q(x) y+\lambda r(x) y=0, & a<x<b\\
                    y(0)=y(T),&\\
                     y'(0)=y'(T), &
        \end{array}
 \right.
\end{empheq}
  \end{frame}
 

 \begin{frame}{Problemas singulares de Sturm-Liouville}
\begin{block}{Propiedades problema periódico}
 Supongamos que $p,q,r$ son funciones continuas y que $p(0)=p(T)$, $q(0)=q(T)$, y $r(0)=r(T)$. Sea $V$ el espacio vectorial de la funciones continuas que satisfacen las condiciones de controno periódicas y $L[u]=(pu')'+qu$. 
 \begin{enumerate}
  \item Demostrar que $L$ es hermitiano en $V$.
  \item A partir de allí demostrar que los autovalores son reales y las autofunciones se pueden tomar a valores reales.
  \item ¿Serán simples los autovalores?
 \end{enumerate}
\end{block}

\textbf{Observacón} Las afirmaciones sobre que los autovalores forman una sucesión y  sobre desarrollos en serie de autofunciones continuan siendo ciertos. 

 
 
\end{frame}

 
 \section{Problemas singulares de Sturm-Liouville}
 
 \subsection{Definición y ejemplos}
\begin{frame}{Problemas singulares de Sturm-Liouville}
Supongamos el operador
$$L[y](x):=\frac{d}{d x}\left[p(x) \frac{d y}{d x}\right]+q(x) y$$
y consideremos la ecuación
 $$\quad L[y](x)+\lambda r(x) y(x)=0, \quad a<x<b.$$
donde  $p(x), p^{\prime}(x), q(x)$ y $r(x)$ son  continuas en $(a, b)$ con valores reales, y además $p(x)$ y $r(x)$ son positivas en $(a, b)$. Llamamos a la ecuación  una ecuación \emph{singular de Sturm-Liouville}  si una  de las siguientes situaciones ocurren:

\begin{itemize}
 \item<+-> $\lim _{x \to a^{+}} p(x)=0 \circ \lim _{x\to b^-}(x)=0$.
 \item<+-> $p(x), q(x)$ o $r(x)$ se vuelven no acotadas cuando $x$ tiende a $a$ o $b$.
 \item<+-> El intervalo $(a, b)$ no está acotado (por ejemplo, $(a, \infty),(-\infty, b)$ o $(-\infty, \infty))$.
\end{itemize}


\end{frame}



 
  
\begin{frame}{Ejemplos: Ecuación Bessel}

La \emph{ecuación de Bessel de orden $\nu$}
\begin{empheq}[box=\tcbhighmath]{equation}\label{eq:bessel}  
 t^{2} y^{\prime \prime}(t)+ty^{\prime}(t)+\left(t^{2}-\nu^{2}\right) y=0, \quad 0<t<\sqrt{\lambda} b
\end{empheq}
 
se puede transformar mediante la sustitución $t=\sqrt{\lambda} x$ en la ecuación singular de Sturm-Liouville

\begin{empheq}[box=\tcbhighmath]{equation}\label{eq:bessel2}  
\frac{d}{d x}\left[x \frac{d y}{d x}\right]-\frac{\nu^{2}}{x} y+\lambda x y=0, \quad 0<x<b.
\end{empheq}


En este caso, $p(x)=x$, que se anula para $x=0$. Además, para $\nu \neq 0$, la función $q(x)=-\nu^{2} / x$ no está acotada cuando $x \rightarrow 0^{+}$.

\end{frame}



 
  
\begin{frame}{Ejemplos: Ecuación de Legendre}

\begin{empheq}[box=\tcbhighmath]{equation}\label{eq:legendre}  \left(1-x^{2}\right) y^{\prime \prime}-2 x y^{\prime}+n(n+1) y=0, \quad-1<x<1
\end{empheq}

se puede escribir como la ecuación singular de Sturm-Liouville


\begin{empheq}[box=\tcbhighmath]{equation}\label{eq:legendre2} 
\frac{d}{d x}\left[\left(1-x^{2}\right) \frac{d y}{d x}\right]+\lambda y=0, \quad-1<x<1,
\end{empheq}
donde $\lambda=n(n+1)$. En este caso, $p(x)=1-x^{2}$ se anula en los dos extremos $\pm 1$.

\end{frame}



 
  
\begin{frame}{Ejemplos: Ecuación de Hermite}

\begin{empheq}[box=\tcbhighmath]{equation}\label{eq:hermite1}  
 y^{\prime \prime}-2 x y^{\prime}+2 n y=0, \quad-\infty<x<\infty,
\end{empheq}
al ser multiplicada por $e^{-x^{2}}$, se convierte en la ecuación singular de Sturm-Liouville



\begin{empheq}[box=\tcbhighmath]{equation}\label{eq:hermite1}  
 \frac{d}{d x}\left[e^{-x^{2}} \frac{d y}{d x}\right]+\lambda e^{-x^{2}} y=0, \quad-\infty<x<\infty,
\end{empheq}
donde $\lambda=2 n$. En este caso, el intervalo $(-\infty, \infty)$ no está acotado.



\end{frame}



 
\subsection{Autoadjunción}
\begin{frame}{Autoadjunción}

¿Cuáles condiciones en la frontera convierten a una ecuación singular de Sturm-Liouville en un problema autoadjunto?

Debemos tener
$$
\int_{a}^{b}(u L[v]-v L[u])(x) d x=\lim _{x \rightarrow b} p(x) W[u, v](x)-\lim _{x \rightarrow a^{+}} p(x) W[ u, v](x) .
$$
 
\begin{empheq}[box=\tcbhighmath]{multline}\label{eq:cond_suf}  
\lim _{x \rightarrow a^{+}} p(x)\left[u(x) v^{\prime}(x)-u^{\prime}(x) v(x)\right]\\=\lim _{x \rightarrow b^{-}} p(x)\left[u(x) v^{\prime}(x)-u^{\prime}(x) v(x)\right]=0
\end{empheq}

\end{frame}



 
  
\begin{frame}{Condiciones singulares en la frontera}



\begin{block}{Lema} Cualquiera de las  condiciones siguientes garantizan la ecuación \eqref{eq:cond_suf} en $a$:
    \begin{enumerate}
        \item<+-> $\lim _{x \rightarrow a^{+}} p(x)=p(a)$ existe y $u, v$ satisfacen la condición en la frontera
            \begin{equation}\label{eq:con_cont_1}
            a_{1} y(a)+a_{2} y^{\prime}(a)=0,\hbox{ con  }a_{1}, a_{2} \hbox{ no ambos nulos.}
            \end{equation}

        \item<+->$\lim_{x\to a^+}p(x)=0$ y $u, v$ satisfacen la condición en la frontera
            \begin{equation}\label{eq:con_cont_2}
                y(x), y^{\prime}(x) \hbox{ permanecen acotados cuando } x \rightarrow a^{+}.
            \end{equation} 
        \item<+-> Las funciones $u, v$ satisfacen la condición en la frontera
            \begin{equation}\label{eq:con_cont_3}
                \lim _{x \rightarrow a^{+}} \sqrt{p(x)} y(x)=0 \hbox{ y }\lim _{x \rightarrow a^{+}} \sqrt{p(x)} y^{\prime}(x)=0
            \end{equation}
       \end{enumerate}
\end{block}

\end{frame}



 
  
\begin{frame}{Condiciones singulares en la frontera}

\onslide<+->
\textbf{Demostración.}

(1) Sin pérdida de generalidad, suponemos que $a_{1} \neq 0$, Entonces,  implica que $u(a)=-\left(a_{2} / a_{1}\right) u^{\prime}(a)$ y $v(a)=-\left(a_{2} / a_{1}\right) v^{\prime}(a)$. Al sustituir

$$
\begin{aligned}
p(a)\left[u(a) v^{\prime}(a)-u^{\prime}(a) v(a)\right] &=p(a)\left[-\left(a_{2} / a_{1}\right) u^{\prime}(a) v^{\prime}(a)+\left(a_{2} / a_{1}\right) u^{\prime}(a) v^{\prime}(a)\right] \\
&=0
\end{aligned}
$$

\onslide<+->
(2)  Si $u, u^{\prime}, v$ y $ v^{\prime}$ permanecen acotadas cuando $x$ tiende a $a$ por la derecha, entonces  $W[u,v]$ permanece acotado. Como $\lim_{x \rightarrow a^{+}} p(x)=0$ y el producto de un factor acotado, por un factor que tiende a cero tiende a cero,  se cumple \eqref{eq:cond_suf}.

\end{frame}



 
  
\begin{frame}{Condiciones singulares en la frontera}

(3) Si $u$ y $v$ satisfacen \eqref{eq:con_cont_3}, entonces 
$$
\begin{aligned}
\lim _{x \rightarrow a^{+}} p(x)\left[u(x) v^{\prime}(x)-u^{\prime}(x) v(x)\right]=&\left(\lim _{x \rightarrow a^{+}} \sqrt{p(x)} u(x)\right)\left(\lim _{x \rightarrow a^{\prime}} \sqrt{p(x)} v^{\prime}(x)\right) \\
&-\left(\lim _{x \rightarrow a^{+}} \sqrt{p(x)} u^{\prime}(x)\right)\left(\lim _{x \rightarrow a^{+}} \sqrt{p(x)} v(x)\right) \\
=& 0-0=0 .
\end{aligned}
$$
Por lo tanto, se cumple la ecuación \eqref{eq:cond_suf}.

\onslide<+->
Se puede establecer  similarmente las mismas afirmaciones para el punto $b$ 

 \end{frame}

 \subsection{ Ecuación de Bessel}
 
\begin{frame}{Ejemplo: Ecuación de Bessel}
 \onslide<+->
Un problema singular de Sturm-Liouville con valores en la frontera típico asociado a la ecuación de Bessel de orden $\nu$ es

\begin{empheq}[box=\tcbhighmath,left=\left\{,right=\right.]{align}  
        \frac{d}{d x}\left[x \frac{d y}{d x}\right]-\frac{\nu^{2}}{x} y+\lambda x y=0, \quad 0<x<1.\\
        y(x), y^{\prime}(x) \hbox{ permanecen acotadas cuando } x \rightarrow 0^{+},\\
        y(1)=0.
\end{empheq}

\onslide<+->
Este problema es autoadjunto, pues la condición (2) del lema  se cumple en $a=0$ y  la condición (1)  se cumple en $b=1$.

Si $\lambda>0$, la sustitución $t=\sqrt{\lambda} x$ transforma la  ecuación en la ecuación de  Bessel 
\begin{empheq}[box=\tcbhighmath]{multline*}  
t^{2} y^{\prime \prime}+t y^{\prime}+\left(t^{2}-\nu^{2}\right) y=0.
\end{empheq}
 \end{frame}

\begin{frame}{Ejemplo: Ecuación de Bessel}
\onslide<+->
En cursos básicos de ecuaciones diferenciales se demuestra que la solución general se escribe
\begin{empheq}[box=\tcbhighmath]{multline*}  
y(t)=c_{1} J_{\nu}(t)+c_{2} Y_{\nu}(t)
\end{empheq}
o
$$
y(x)=c_{1} J_{\nu}(\sqrt{\lambda} x)+c_{2} Y_{\nu}(\sqrt{\lambda} x)
$$

\onslide<+->

\begin{itemize}
 \item<+-> $J_{\nu}$ se llama \emph{Función de Bessel de primera especie} y satisface las hipótesis (2) del Lema en $x=0$ para $\nu=0$ y $\nu\geq 1$.
 \item<+-> $Y_{\nu}$ se llama \emph{Función de Bessel de segunda especie} y no está acotada cerca de $x=0$.
\end{itemize}

 \end{frame}

\begin{frame}{Ejemplo: Ecuación de Bessel}

\onslide<+->
Para que se cumpla  la condición de contorno en $x=0$ debemos tener $c_{2}=0$ 

Para que se cumpla la condición de contorno en $x=1$  con $y(x)=c_{1} J_{\nu}(\sqrt{\lambda} x)$, necesitamos que $c_{1}=0$ o 
$$
J_{\nu}(\sqrt{\lambda} )=0
$$
 
 \onslide<+->
 
\begin{block}{Teorema} 
La función de Bessel $J_{\nu}$ tiene una sucesión creciente de ceros reales:
$$
0<\alpha_{\nu 1}<\alpha_{\nu 2}<\alpha_{\nu 3}<\cdots
$$
\end{block}

 \end{frame}

\begin{frame}{Ejemplo: Ecuación de Bessel}
\onslide<+->
Por lo tanto, 

$$\lambda_{\nu n}=\alpha_{\nu n}{ }^{2}, \nu,n=0,1,2,3, \ldots$$
 son valores propios  con funciones propias correspondientes
$$y_{\nu n}(x)=c_{1} J_{\nu}\left(\alpha_{\nu n} x\right).$$

Éstos son los únicos valores propios positivos. Además, se puede mostrar que no existen valores propios no positivos.
\onslide<+->


Las funciones propias en  son ortogonales con respecto de la función de ponderación $r(x)=x$ :

\begin{empheq}[box=\tcbhighmath]{multline*}  
\int_{0}^{1} J_{\nu}\left(\alpha_{\nu n} x\right) J_{\nu}\left(\alpha_{\nu m} x\right) x d x=0, \quad n \neq m
\end{empheq}

\end{frame}


\begin{frame}{Ejemplo: Ecuación de Bessel}
\onslide<+->

Si $f(x)$ es una función dada, entonces un desarrollo mediante funciones propias asociado a $f$ es
\begin{empheq}[box=\tcbhighmath]{multline*}  
f \sim \sum_{n=1}^{\infty} a_{n} J_{\nu}\left(\alpha_{v n} x\right)
\end{empheq}

donde
\begin{empheq}[box=\tcbhighmath]{multline*}  
a_{n}=\frac{\int_{0}^{1} f(x) J_{\nu}\left(\alpha_{\nu n} x\right) x d x}{\int_{0}^{1} J_{v}^{2}\left(\alpha_{\nu n} x\right) x d x}, \quad n=1,2,3, \ldots
\end{empheq}

\onslide<+->
Cuando $f$ y $f^{\prime}$ son continuas por partes en $[0,1]$, el desarrollo en converge a 
$$\left[f\left(x^{+}\right)+f\left(x^{-}\right)\right] / 2$$ 

para cada $x$ en $(0,1)$.

\end{frame}

\subsection{Ecuación de Legendre} 
\begin{frame}{Ejemplo: Ecuación de Legendre} 
\onslide<+->
Un problema singular de Sturm-Liouville con valores en la frontera típico asociado a la \emph{ecuación de Legendre} es
\onslide<+->
\begin{empheq}[box=\tcbhighmath,left=\left\{,right=\right.]{align}  
            \frac{d}{d x}\left[\left(1-x^{2}\right) \frac{d y}{d x}\right]+\lambda y=0, \quad-1<x<1\\
            y(x), y^{\prime}(x) \hbox{ permanecen acotadas cuando } x \rightarrow \pm 1.
\end{empheq}
\onslide<+->

Es autoadjunto, pues se cumple  la condición (2) del lema  en $x=\pm 1$. 

En cursos de ecuaciones diferenciales ordinarias se ve que la ecuación  tiene soluciones polinomiales para $\lambda=n(n+1), n=0,1,2, \ldots$.  Estas soluciones son múltiplos constantes de los \emph{polinomios de Legendre}
\begin{empheq}[box=\tcbhighmath]{multline*}  
P_{n}(x)=2^{-n} \sum_{m=0}^{[n / 2]} \frac{(-1)^{m}(2 n-2 m) !}{(n-m) ! m !(n-2 m) !} x^{n-2 m}
\end{empheq}
donde $[n / 2]$ es la parte entera.
\end{frame}


\begin{frame}{Ejemplo: Ecuación de Legendre} 
\onslide<+->
  $P_{n}(x)$ son ortogonales en $[-1,1]$ con respecto de $r(x)=1$ :
 
\begin{empheq}[box=\tcbhighmath]{multline*}  
\int_{-1}^{1} P_{n}(x) P_{m}(x) d x=0, \quad n \neq m
\end{empheq}

\onslide<+->

Si $f(x)$ es una función dada, entonces un desarrollo mediante funciones propias para $f(x)$, en términos de los polinomios de Legendre es

\begin{empheq}[box=\tcbhighmath]{multline*}  
 f \sim \sum_{n=0}^{\infty} a_{n} P_{n}(x),
\end{empheq}
 
donde
\begin{empheq}[box=\tcbhighmath]{multline*}  
a_{n}=\frac{\int_{-1}^{1} f(x) P_{n}(x) d x}{\int_{-1}^{1} P_{n}^{2}(x) d x}, \quad n=0,1,2, \ldots
\end{empheq}
\onslide<+->

Los resultados de convergencia similares a los dados para series de Fourier se cumplen para este desarrollo. 

\end{frame}

\subsection{Ecuación de Hermite}
\begin{frame}{Ejemplo: Ecuación de Hermite}

Un problema singular de Sturm-Liouville con valores en la frontera asociado a la ecuación de Hermite es
\begin{empheq}[box=\tcbhighmath,left=\left\{,right=\right.]{align}  
     \frac{d}{d x}\left[e^{-x^{2}} \frac{d y}{d x}\right]+\lambda e^{-x^{2}} y=0, \quad-\infty<x<\infty\\
    \lim _{x \rightarrow+\infty} e^{-x^{2} / 2} y(x)=0 \hbox{ y } \lim _{x \rightarrow+\infty} e^{-x^{2} / 2} y^{\prime}(x)=0.
\end{empheq}
Por la condición (3) del lema , el operador lineal asociado al problema es autoadjunto.


Se puede demostrar que se puede transformar la ecuación del problema en  la ecuación de Hermite 

$$y^{\prime \prime}-2 x y^{\prime}+2 n y=0.$$

También que esta ecuación tiene soluciones polinomiales que son múltiplos constantes de los \emph{polinomios de Hermite $H_{n}(x), n=0,1,2, \ldots$ 
}\end{frame}


\begin{frame}{Ejemplo: Ecuación de Hermite}
Las soluciones polinomiales son las únicas que satisfacen las condiciones en la frontera. Se deduce que los valores propios son $\lambda_{n}=2 n, n=0,1,2, \ldots$, con funciones propias correspondientes $H_{n}(x)$. 


Los polinomios de Hermite $H_{n}(x)$ son ortogonales en $(-\infty, \infty)$ con respecto de la función de ponderación $r(x)=e^{-x^{2}}$ :

\begin{empheq}[box=\tcbhighmath]{equation}  
    \int_{-\infty}^{\infty} H_{n}(x) H_{m}(x) e^{-x^{2}} d x=0, \quad n \neq m.
\end{empheq}

\end{frame}


\begin{frame}{Ejemplo: Ecuación de Hermite}
Si $f$ es una función dada, entonces un desarrollo con funciones propias asociado a $f$ está dado por
$$f \sim \sum_{n=0}^{\infty} a_{n} H_{n}(x)$$ 
donde
$$
a_{n}=\frac{\int_{-\infty}^{\infty} f(x) H_{n}(x) e^{-x^{2}} d x}{\int_{-\infty}^{\infty} H_{n}^{2}(x) e^{-x^{2}} d x}, \quad n=0,1,2, \ldots
$$

 \end{frame}

 \subsection{Aplicaciones}
 


\begin{frame}{Vibraciones transversales de una membrana elástica}
Membrana elástica, por ejemplo un tambor, que es mantenida fija a un arocircular. 
Viene gobernada por la ecuación de \emph{ondas bidimensional}

\[ u_{tt}=\Delta u:=u_{xx}+u_{yy}\]

\begin{figure}
    \begin{center}
        \animategraphics[autoplay , scale=.25,loop=true]{15}{/home/fernando/fer/Docencia/grado/EcuacionesDiferenciales/Materiales/Teoria_Basica/membrana/12/mem-}{0}{49}
    \end{center}  
\caption{Membrana vibrante}
\end{figure}
 \end{frame}

\begin{frame}{Variables y suposiciones}


\textbf{Variables}

\begin{itemize}
 \item<+->$t$ es el tiempo,
 \item<+-> $x,y,u$ son las coordenadas de un punto sobre la membrana en un sistema de coordenadas cartesiano ortogonal.

\end{itemize}

\onslide<+->

\textbf{Suposiciones.}
\begin{itemize}
 \item<+-> No actúa otra fuerza más que la tensión de la membrana.
 \item<+->El material de esta membrana es uniforme.
 \item<+->La dirección de desplazamientos de un punto sobre la membrana es perpendiculares al plano que contiene al aro de sujeción. 
 \item<+-> El aro de sujeción  se supone de radio $1$ y centro en $(0,0)$ y está contenido en el plano $x,y$. 
 \item<+-> $B$ denota la bola de centro $(0,0)$ y radio $1$.  
 
 \end{itemize}


 \end{frame}
 

\begin{frame}{Condiciones de contorno e iniciales}
\onslide<+->
\emph{Condición de contorno}
\begin{empheq}[box=\tcbhighmath ]{equation}  
    u=0 \hbox{ en } \partial B\quad\hbox{ membrana fija en el aro}.
\end{empheq}
\onslide<+->

     \emph{Condiciones iniciales} nos dicen cual es el estado de la membrana en $t=0$, 
     
     $$ \left\{
                \begin{array}{ll}
                    u(x,y,0)=f(x) & x\in B,\\
                     u_t(x,y,0)=g(x)& x\in B.\\
                \end{array}
                \right.
$$
\onslide<+->

Supondremos, para simplificar los cálculos, $g\equiv 0$

\begin{empheq}[box=\tcbhighmath,left=\left\{,right=\right.]{align}       
                    u(x,y,0)&=f(x),&(x,y)\in B,\label{eq:cod_ini_1}\\
                    u_t(x,y,0)&=0,&(x,y) \in B.\label{eq:cod_ini_2}
\end{empheq} 


  \end{frame}
 


 

\begin{frame}{Separación variables}

\onslide<+->
Reemplazando $u(x,y,t)=v(x,y)T(t)$ en la ecuación:

\begin{equation}\label{eq:sep_var}v(x,y)T''(t)=T(t)\left(v_{xx}(x,y)+v_{yy}(x,y) \right).
\end{equation}
\onslide<+->
Vale decir 
\[\frac{T''(t)}{T(t) }=\frac{v_{xx}(x,y)+v_{yy}(x,y)}{v(x,y)}.\]
\onslide<+->
Esto implica  $T''/T$  y  $\Delta v/v$ son constantes. Existe $\lambda$ tal que
 \[\frac{T''(t)}{T(t) }=\frac{v_{xx}(x,y)+v_{yy}(x,y)}{v(x,y)}=-\lambda.\]
\onslide<+->
 Tenemos así  dos ecuaciones
\begin{empheq}[box=\tcbhighmath,left=\left\{,right=\right.]{align}  
  T''(t)+\lambda T(t)=0, & t\in\rr\label{eq:eq_t} \\
  v_{xx}(x,y)+v_{yy}(x,y)+\lambda v(x,y)=0, & (x,y)\in B\label{eq:eq_xy}
\end{empheq}
  \end{frame}
 
\begin{frame}{Positividad del autovalor}

\begin{block}{Ejercicio}
a) Sean $u,v\in C^1(\Omega)\cap C(\overline{\Omega})$.  Aplicar ejercicio de identidad de Green a) con $M=uv$ y $N=0$ y deducir la  \emph{fórmula de integración por partes}
 \[
  \iint_{\Omega}uv_xdxdy=-\iint_{\Omega}u_xvdxdy+\oint uvn_xds,
 \]
donde $\vec{n}=(n_x,n_y)$ es el vector normal exterior a $\partial \Omega$. Por supuesto vale una fórmula similar con la derivada respecto a $y$.

b) Suponer $v\not\equiv 0$. Multiplicar la ecuación \eqref{eq:eq_xy} por  $v$ la segunda ecuación en \eqref{eq:sep_var} integrar en $\Omega$  y usar la fórmula del inciso a) para deducir que 
\[
 \lambda=\frac{\displaystyle\iint_{\Omega}v_x^2+v_x^2dxdy}{\displaystyle\iint_{\Omega}v^2dxdy}>0
\]


\end{block}

 
  \end{frame}
 
 
\begin{frame}{condiciones de contorno  }


\onslide<+->
Las condiciones de contorno  

$$u(x,y,t)=0, \quad (x,y)\in \partial B\Rightarrow \tikzmarkin<2->{a1} v=0\hbox{ en } \partial B\tikzmarkend{a1}
$$

\onslide<+->
La condición inicial \eqref{eq:cod_ini_2} implica
\[
 u_t(x,y,0)=v(x,y)T'(0)=0\Rightarrow\tikzmarkin<3->{b1} T'(0)=0\hbox{ en } \partial B\tikzmarkend{b1}
\]
\onslide<4->
No es posible satisfacer las condiciones iniciales  con la función propuesta, pues
\[
 u(x,y,0)=v(x,y)T(0)=f(x,y).
\]
lo se cumpliría si, por casualidad, elegimos $f$ solución de 
$$v_{xx}(x,y)+v_{yy}(x,y)+\lambda v(x,y)=0.$$ 

Más adelante veremos como tratar las condiciones iniciales.
  \end{frame}
 
 
\begin{frame}{Resolviendo}
\onslide<+->
Escribamos $\lambda=\omega^2$.  La solución general de la ecuación \eqref{eq:eq_t} con la condición $T'(0)=0$ es

\begin{empheq}[box=\tcbhighmath]{equation}\label{eq:sol_T}
    T(t)=k\cos(\omega t),\quad k\in\rr.
\end{empheq}

\onslide<+->
La ecuación \eqref{eq:eq_xy} se conoce como la ecuación de autovalores del Laplaciano o \emph{ecuación de Helmholtz}.  Los valores de $\lambda$ para los que esta ecuación tiene solución se llaman  autovalores del operador de Laplace.

\onslide<+->

 Para encontrar estos autovalores vamos a usar coordenadas polares $v=v(r,\theta)$. Escribiendo el Laplaciano en estas coordenadas 

 \begin{equation}\label{eq:ecu_aux_1}v_{rr}+\frac{1}{r}v_r+\frac{1}{r^2}v_{\theta\theta}+\omega^2v=0
\end{equation}

  \end{frame}
 
 
\begin{frame}{Separando variables otra vez}
Nuevamente vamos a considerar la técnica de separación de variables. Proponemos que
\[v(r,\theta)=R(r)\Theta(\theta)\]

Reemplazando en \eqref{eq:ecu_aux_1}
\begin{equation*}\label{eq:ecua_aux_2} 
R''\Theta +\frac{1}{r}R'\Theta+\frac{1}{r^2}R\Theta''+\omega^2R\Theta=0.
\end{equation*}
Multiplicando por $r^2/R\Theta$ y depejando los términos conteniendo $\Theta$
\begin{equation}\label{eq:ecua_aux_3} 
\frac{r^2R''}{R} +\frac{rR'}{R}+\omega^2r^2=-\frac{\Theta''}{\Theta}.
\end{equation}
  \end{frame}

 
 
\begin{frame}{Resolviendo}
\onslide<+->
 Cada miembro depende de variables independientes entre si, por consiguiente las funciones deben ser constantes. Debe existir $\mu$ tal que
\begin{align}
r^2R'' +rR'+(\omega^2r^2-\mu) R&=0\label{eq:ecua_aux_4}\\
\Theta'' +\mu\Theta=0\label{eq:ecua_aux_5}.
%\end{split}
\end{align}
Además la condición de contorno implica
\begin{equation}\label{eq:cond_contor2}
R(1)=0.
\end{equation}

Queremos una solución acotada cuando $r\to 0^+$, esto lo escribimos 

\begin{equation}\label{eq:cond_contor3}
|R(0)|<\infty.
\end{equation}

\onslide<+->
La función $\Theta$ al depender de $\theta$ debería ser periódica. Para que esto sea así  $\mu$ debe ser positivo, puesto que las soluciones de  \ref{eq:ecua_aux_4} son periódicas solo para estos valores de $\mu$. Por consiguiente
\begin{equation}\label{sol_theta}
\Theta(\theta)=c_1\cos\sqrt{\mu}\theta+c_2\sen\sqrt{\mu}\theta.
\end{equation}

  \end{frame}

 
 
\begin{frame}{Resolviendo}

 Como   más específicamente el  período debe ser $2\pi$, el valor de $\mu$ debe ser un entero cuadrado, es decir que existe un entero positivo  $n$ tal que $\mu=n^2$.   Así \eqref{sol_theta} se convierte en 
\[\Theta(\theta)= c_1\cos n\theta+c_2\sen n\theta.\]

Obtendremos dos soluciones linealmente independientes eligiendo $c_1=0$ y $c_2=1$ y permutando estos valores. 

\begin{empheq}[box=\tcbhighmath]{equation}\label{sol_theta2}
 \Theta_{1n}(\theta)= \cos n\theta, \quad \Theta_{2n}(\theta)= \sen n\theta.
\end{empheq}


  \end{frame}

 
 
\begin{frame}{Resolviendo}



 Reemplazando $\mu$ por $n^2$ en \eqref{eq:ecua_aux_4} 
\begin{equation}\label{eq:ecua_aux_6}r^2R'' +rR'+(\omega^2r^2-n^2) R=0,\end{equation}

La podemos convertir facilmente en la ecuación de Bessel por el cambio de variable independiente $s=\omega r$. Tenemos 

\[\frac{dR}{dr}=\frac{dR}{ds}\frac{ds}{dr}=\frac{dR}{ds}\omega\]
y
\[\frac{d^2R}{dr^2}=\frac{d^2R}{ds^2}\omega^2\]
Reemplazando las igualdades anteriores y $r$ por $s/\omega$ en \eqref{eq:ecua_aux_6} llegamos a
\[s^2R''(s)+sR'+(s^2-n^2)R=0\]

  \end{frame}

 
 
\begin{frame}{Resolviendo}
Obtuvimos la ecuación de Bessel de orden $n$, con $n$ entero no negativo. La solución general se escribe:
$$R=c_1J_n+c_2Y_n$$

donde  $J_n$  es continua en $0$ e $Y_n$  es no acotada en $0$. Pero \eqref{eq:cond_contor3} $\Rightarrow c_2\neq 0$. Sin perder generalidad, supongamos $c_1=1$. Así tenemos que $R$ como función de $r$ es

\begin{empheq}[box=\tcbhighmath]{equation}\label{eq:sol_R}
 R(r)=J_n(\omega r).
 \end{empheq}
   \end{frame}

 
 
\begin{frame}{Resolviendo}
Ahora la condición de contorno \eqref{eq:ecua_aux_5} implica que

\begin{empheq}[box=\tcbhighmath]{equation}\label{eq:cer_bessel}
J_n(\omega)=0.
 \end{empheq}


Sabemos que los ceros de la ecuación de Bessel forman una sucesión 

$$\omega_{n0}<\omega_{n1}<\cdots,\quad \text{ con } \omega_{nk}\nearrow\infty, \text{ cuando }k\to\infty$$ 

Tenemos una solución distinta por cada $n$ entero positivo y por cada $\omega_{nk}$ en la lista de ceros de $J_n$

\begin{empheq}[box=\tcbhighmath]{align}
 u_{1nk}(r,\theta,t)&=\cos(\omega_{nk} t)v_{1nk}(r,\theta)&=\cos(\omega_{nk} t)\cos(n\theta)J_n(\omega_kr)\label{eq:tono_normal1}\\
  u_{2nk}(r,\theta,t)&=\cos(\omega_{nk} t)v_{2nk}(r,\theta)&=\cos(\omega_{nk} t)\sen(n\theta)J_n(\omega_{nk}r)\label{eq:tono_normal2}
\end{empheq}

   \end{frame}

 
 
\begin{frame}{Tonos normales}

Las funciones $u_{1nk},u_{2nk}$ se llaman \emph{tonos normales}. Todos los puntos de la membrana vibran a la misma frecuencia $\omega_{nk}$. Si es un tambor se produce una nota pura.

\begin{tabular}{ccc}
 \animategraphics[autoplay , scale=.2,loop=true]{15}{/home/fernando/fer/Docencia/grado/EcuacionesDiferenciales/Materiales/Teoria_Basica/membrana/01/mem-}{0}{49}
 &
 \animategraphics[autoplay , scale=.2,loop=true]{15}{/home/fernando/fer/Docencia/grado/EcuacionesDiferenciales/Materiales/Teoria_Basica/membrana/02/mem-}{0}{49}
 &
 \animategraphics[autoplay , scale=.2,loop=true]{15}{/home/fernando/fer/Docencia/grado/EcuacionesDiferenciales/Materiales/Teoria_Basica/membrana/12/mem-}{0}{49}
\\
$n=0,k=0$ & $n=0,k=1$ & $n=1,k=1$  
\end{tabular}

    \end{frame}
    
    

 
 
\begin{frame}{Ortogonalidad} 

\begin{block}{Ejercicio} 
La familia de funciones $\{v_{1nk}\}_{n,k=0}^\infty\cup \{v_{2nk}\}_{n,k=0}^\infty$ es ortogonal en $B$ con ponderación $r$. Es decir si $(n,k)\neq(n',k')$
\begin{empheq}[box=\tcbhighmath]{equation}
 \int_0^{2\pi}\int_0^1\cos(n\theta)\cos(n'\theta)J_n(\omega_kr) J_{n'}(\omega_{k'}r)r d\theta dr=0 
\end{empheq}
y $\forall (n,k),(n',k')$
\begin{empheq}[box=\tcbhighmath]{equation}
 \int_0^{2\pi}\int_0^1\cos(n\theta)\sen(n'\theta)J_n(\omega_kr) J_{n'}(\omega_{k'}r)r d\theta dr=0 
\end{empheq}

 
\end{block}


   \end{frame}
   
   
\begin{frame}{Desarrollo en serie, completitud} 

\begin{block}{Teorema}

La familia de funciones $\{v_{1nk}\}_{n,k=0}^\infty\cup \{v_{2nk}\}_{n,k=0}^\infty$ es un sistema  ortogonal completo  en $B$ con ponderación $r$. Si $f:B\to\rr$ es una función continua de cuadrado integrable  entonces

\begin{empheq}[box=\tcbhighmath]{equation}
f(r, \theta)= \sum_{n=0}^{\infty} \sum_{k=0}^{\infty} J_{n}\left(\omega_{k m} r\right)\left(a_{nk} \cos n\theta +b_{nk} \sen n \theta\right)
\end{empheq}
 
Con convergencia en media. Los coeficientes vienen dados por

\begin{empheq}[box=\tcbhighmath]{align}
    a_{nk}&=\frac{2}{\pi J_{n+1}(\omega_{nk})} \int_{0}^{2 \pi} \int_{0}^{1} f(r, \theta) J_{n}(\omega_{nk } r) \cos n\theta  d r d \theta\\
    b_{nk}&=\frac{2}{\pi J_{n+1}(\omega_{nk})} \int_{0}^{2 \pi} \int_{0}^{1} f(r, \theta) J_{n}(\omega_{nk } r) \sen n\theta  d r d \theta
\end{empheq}

 
\end{block}

   \end{frame}
   
   
\begin{frame}{Separación variables} 

Finalmente se propone una solución al problema original de la forma
\begin{empheq}[box=\tcbhighmath]{equation}
u(r,\theta,t)=\sum_{n=0}^{\infty} \sum_{k=0}^{\infty} \cos(\omega_{nk} t)J_{n}\left(\omega_{k m} r\right)\left(a_{nk} \cos n\theta +b_{nk} \sen n \theta\right)
\end{empheq}


Esta $u$ será solución pues es una suma de soluciones, satisface que $u=0$ en $\partial B$ y que $u_t(x,y,0)=0$. Debería cumplirse

$$f(x,y)=u(x,y,0),$$
y el Teorema anterior nos dice como elegir $a_{nk},b_{nk}$ para conseguir esto. 
   \end{frame}
   

   
\end{document}
