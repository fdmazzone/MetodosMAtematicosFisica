

% Inicio de la presentacion
% Para ejecutar el Python
% /usr/share/texlive/texmf-dist/scripts/pythontex/pythontex3.py EcuacionesLineales.tex
%\documentclass[handout]{beamer}
\documentclass[xcolor=dvipsnames,a4paper,10pt,handout]{beamer}
	\mode<presentation>
\usetheme{Warsaw}
\usecolortheme[named=Salmon]{structure}%rosa


\newcommand{\rr}{\mathbb{R}}
\renewcommand{\emph}[1]{\textcolor{blue}{#1}}
\renewcommand{\textbf}[1]{\textcolor{green}{\bfseries #1}}



\setbeamertemplate{footline}
{
  \hbox{\begin{beamercolorbox}[wd=1\paperwidth,ht=2.25ex,dp=1ex,right]{framenumber}%
      \usebeamerfont{framenumber}\insertframenumber{} / \inserttotalframenumber\hspace*{2ex}
    \end{beamercolorbox}}%
  \vskip0pt%
}
%\definecolor{beamer@blendedblue}{rgb}{0.137,0.466,0.741}
\definecolor{gris}{HTML}{7adcd2}
\definecolor{rojo}{HTML}{f2c08f}
\definecolor{frametit}{HTML}{fd779d}
% 
% \setbeamercolor{structure}{fg=beamer@blendedblue}
% \setbeamercolor{titlelike}{parent=structure}
% \setbeamercolor{frametitle}{fg=black}
% \setbeamercolor{title}{fg=black}
% \setbeamercolor{item}{fg=black}

 


%\usecolortheme{}
%\usecolortheme[named=Aquamarine]{structure}
%\usetheme{Singapore}%{Marburg}%{Berkeley}%{Antibes}%{Darmstadt}
%\usefonttheme{serif}


% 
% \setbeamertemplate{navigation symbols}{}
% \setbeamertemplate{footline}{}
% 
% Personalizar el color de los títulos
% 
% \setbeamercolor{title}{fg=black,bg=Aquamarine}
% \setbeamercolor{block title example}{fg=black,bg=Lavender!90}
% \setbeamercolor{block title alerted}{fg=black,bg=Lavender!80}
% \setbeamercolor{block body alerted}{fg=black,bg=Lavender!20}






\usepackage{tikz}
\usepackage{mathrsfs}
\usepackage[utf8]{inputenc}
\usepackage[spanish]{babel}
\usepackage{beamerthemesplit}
\usepackage{verbatim}
\usepackage{amssymb}
\usepackage{graphicx}
\usepackage{animate}
\usepackage[framemethod=TikZ]{mdframed}
\usepackage{empheq}
\usepackage[breakable,many]{tcolorbox}
\usepackage[beamer,customcolors]{hf-tikz}
\usepackage{hyperref}
\tcbset{nobeforeafter}

\tcbset{highlight math style={enhanced,%<-- needed for the `remember' options
colframe=frametit!100,blue!32,colback=frametit!100,boxsep=0pt}}

\pgfdeclareverticalshading{exersicebackground}{100bp}
  {color(0bp)=(blue!20);color(50bp)=(blue!0)}
\mdfdefinestyle{MiEstilo}{innertopmargin=10pt,linecolor=white!100,%
linewidth=2pt,topline=true,tikzsetting={shading=exersicebackground}}  
% 
% \usetikzlibrary{backgrounds}
% \usetikzlibrary{mindmap,trees}	
% 
% 
% \tikzstyle{every picture}+=[remember picture]

\usepackage{beamerhighlight}


%% libraries for snake paths
\usetikzlibrary{decorations}
\usetikzlibrary{decorations.pathmorphing}

%% library for star node
\usetikzlibrary{shapes.geometric}

%% library for computation of coordinates
\usetikzlibrary{calc}


%  
% %%%%%%%%%%%%%%%%%%%%%%%%%%Nuevos comandos entornos%%%%%%%%%%%%%%%%%%%%%%%%%%%%%%%%
% %%%%%%%%%%%%%%%%%%%%%%%%%%%%%%%%%%%%%%%%%%%%%%%%%%%%%%%%%%%%%%%%%%%%%%%%
\newenvironment{demo}{\noindent\emph{Dem.}}{\hfill\qed \newline\vspace{5pt}}

% \newenvironment{observa}{\noindent\textbf{Observación:}}{}
\newcommand{\com}{\mathbb{C}}
\newcommand{\rr}{\mathbb{R}}
\newcommand{\nn}{\mathbb{N}}
% \renewcommand{\epsilon}{\varepsilon}
\renewcommand{\lim}{\mathop{\rm lím}}
\renewcommand{\inf}{\mathop{\rm ínf}}
\renewcommand{\liminf}{\mathop{\rm líminf}}
\renewcommand{\limsup}{\mathop{\rm límsup}}
\renewcommand{\min}{\mathop{\rm mín}}
\renewcommand{\max}{\mathop{\rm máx}}
\renewcommand{\b}[1]{\boldsymbol{#1}}
% \renewenvironment{frame}[1]{}{}

%%%%%%%%%%%%%%%% Funcion característica %%%%%%%%%%%5555555

\DeclareRobustCommand{\rchi}{{\mathpalette\irchi\relax}}
\newcommand{\irchi}[2]{\raisebox{\depth}{$#1\chi$}} % inner command, used by \rchi
\newcommand{\der}[2]{\frac{\partial #1}{\partial #2}} 
 
%  \definecolor{mycolor}{RGB}{204,179,174}
% 
% \tcbset{highlight math style={enhanced,
%   colframe=red!60!black,colback=mycolor,arc=4pt,boxrule=1pt,
%   drop fuzzy shadow}}

% 
% 
% 


%\renewcommand{\lim}{displaystyle\lim}
\DeclareMathOperator{\atan2}{atan2}
\DeclareMathOperator{\sen}{sen}

\pgfdeclareverticalshading{exersicebackground}{100bp}
  {color(0bp)=(black!40);color(50bp)=(black!0)}

\mdfdefinestyle{MiEstilo}{innertopmargin=10pt,linecolor=white!100,%
linewidth=2pt,topline=true,tikzsetting={shading=exersicebackground}}  



%%%%%%%%%%%%%%%%  Recuadro ecuacion %%%%%%%%%%%%%%%%%%%%%%%%


\newcommand{\boxedeq}[2]{%
\begin{empheq}[box=\tcbhighmath]{equation}\label{#2} #1 \end{empheq}}

%\newcommand{boxedeq}[1]{\textbf{#1}}








%%%%%%%%%%%%%%%%%%%%%%%%%%%%%

%%%%%%%%%%%%%%%%%%%%%%%%%%%%%%
%% Codigo
\newenvironment{codigo}[1][]{%
\mdfsetup{style=MiEstilo%
}
\ifstrempty{#1}
{
\begin{mdframed}[]\relax%
\strut \textbf{Codigo}
}
{
\begin{mdframed}[]\relax%
\strut \textbf{Codigo (#1)}
}}{\end{mdframed}}
%%%%%%%%%%%%%%%%%%%%%%%%%%%%%




%%%%%%%%%%%%%%%%%%%%%Colores

\definecolor{color8}{HTML}{8E87C1}
\definecolor{color2}{rgb}{0.44,0.62,0.42}
\definecolor{color3}{rgb}{0.28, 0.51, .68}
\definecolor{color4}{rgb}{0.29,0.3,0.57}


%%%%%%%%%%%%%%%%%%%%%% Configuracion listing

\lstset{ %
  backgroundcolor=\color{white},   % choose the background color; you must add \usepackage{color} or \usepackage{xcolor}
  basicstyle=\footnotesize,        % the size of the fonts that are used for the code
  breakatwhitespace=false,         % sets if automatic breaks should only happen at whitespace
  breaklines=true,                 % sets automatic line breaking
  captionpos=b,                    % sets the caption-position to bottom
  commentstyle=\color{color2},    % comment style
  deletekeywords={...},            % if you want to delete keywords from the given language
  escapeinside={\%*}{*)},          % if you want to add LaTeX within your code
  extendedchars=true,              % lets you use non-ASCII characters; for 8-bits encodings only, does not work with UTF-8
  frame=single,	                   % adds a frame around the code
  keepspaces=true,                 % keeps spaces in text, useful for keeping indentation of code (possibly needs columns=flexible)
  keywordstyle=\color{blue},       % keyword style
  language=Python,                 % the language of the code
  otherkeywords={symbols,dsolve,solve,Eq, simplify, subs, plot,Function},           % if you want to add more keywords to the set
  numbers=left,                    % where to put the line-numbers; possible values are (none, left, right)
  numbersep=5pt,                   % how far the line-numbers are from the code
  numberstyle=\tiny\color{color3}, % the style that is used for the line-numbers
  rulecolor=\color{black},         % if not set, the frame-color may be changed on line-breaks within not-black text (e.g. comments (green here))
  showspaces=false,                % show spaces everywhere adding particular underscores; it overrides 'showstringspaces'
  showstringspaces=false,          % underline spaces within strings only
  showtabs=false,                  % show tabs within strings adding particular underscores
  stepnumber=2,                    % the step between two line-numbers. If it's 1, each line will be numbered
  stringstyle=\color{color4},     % string literal style
  tabsize=2,	                   % sets default tabsize to 2 spaces
  title=\lstname                   % show the filename of files included with \lstinputlisting; also try caption instead of title
}


% newcommand
%   
% \newcommand{\xb}{\vec{x}}
% \renewcommand{\emph}[1]{\textcolor{blue}{\bfseries #1}}
%\graphicspath{/home/fernando/fer/Docencia/posgrado/Derivadas Parciales Maestría/2022/unidad2-7}

 
\DeclareMathOperator{\sen}{sen}

\DeclareMathOperator{\opL}{\mathscr{L}}
\DeclareMathOperator{\opM}{\mathscr{M}}
\DeclareMathOperator{\dive}{div}

\title[Ecuaciones en Derivas Parciales-Aplicaciones] % (optional, nur bei langen Titeln nötig)
{%
Problemas de Sturm-Liouville
}

\author[] % (optional, nur bei vielen Autoren)
{ }

\institute[Problemas de Sturm-Liouville] % (optional, aber oft nötig)
{
 Depto de Matemática\\
Facultad de Ciencias Exactas Físico-Químicas y Naturales\\
Universidad Nacional de Río Cuarto}


\subject{Ecuaciones en Derivadas Parciales}


% 





% 
\begin{document}
% 

\hfsetfillcolor{frametit}
\hfsetbordercolor{frametit}

 
\section{Problema no homogéneo}

\subsection{Bibliografía}
\begin{frame}{Bibliografía}
\begin{center}
\includegraphics[scale=.5]{/home/fernando/fer/Biblioteca/calibre/Dean G. Duffy/Green's Functions With Applications (732)/cover.jpg}
\includegraphics[scale=.22]{/home/fernando/fer/Biblioteca/calibre/Juan M. Aguirregabiria/Ecuaciones diferenciales ordinarias para estudiantes de fisica (313)/cover.jpg}

\end{center}
\end{frame}



\subsection{Desarrollo en autofunciones}
\begin{frame}{Problema regular no homogéneo de Sturm-Liouville}



\begin{empheq}[box=\tcbhighmath,left=\left\{,right=\right.]{equation}\label{eq:no-homogeneo}
    \begin{split}
        &L[u]+\mu r u=f(x),\quad x\in (a,b)\\
        &B[u,a]:=a_1u(a)+a_2u'(a)=0\\
        &B[u,b]:=b_1u(b)+b_2u'(b)=0
    \end{split}
\end{empheq} 

\begin{enumerate}
 \item<+->$\left\{\lambda_n\right\}_{n=1}^{\infty}$ los valores propios con funciones propias correspondientes $\left\{\phi_n\right\}_{n=1}^{\infty}$, asumidas ortonormales.
 \item<+-> Queremos determinar los coeficientes $c_n$ de modo que 
 $$\phi(x)=\sum_{n=1}^{\infty} c_n \phi_n(x)$$
 sea una solución.
\end{enumerate}


\textbf{Observación.} Como cada $\phi_n$ satisface las condiciones en la frontera, también lo hace $\phi$. 
\end{frame}


\begin{frame}{Problema no homogéneo de Sturm-Liouville}

Sustituyendo  el desarrollo en la ecuación y usando  que $L\left[\phi_n\right]=-\lambda_n r \phi_n$, 


\begin{multline*} L[\phi]+\mu r \phi=L\left[\sum_{n=1}^{\infty} c_n \phi_n\right]+\mu r \sum_{n=1}^{\infty} c_n \phi_n
=\sum_{n=1}^{\infty} c_n L\left[\phi_n\right]+\mu r \sum_{n=1}^{\infty} c_n \phi_n\\
=\sum_{n=1}^{\infty} c_n\left(-\lambda_n r \phi_n\right)+\mu r \sum_{n=1}^{\infty} c_n \phi_n 
=r \sum_{n=1}^{\infty}\left(\mu-\lambda_n\right) c_n \phi_n .
\end{multline*}

Desarrollamos $(f / r)$ mediante funciones propias
$$
f / r=\sum_{n=1}^{\infty} \gamma_n \phi_n, \quad \gamma_n=\ \int_a^b(f / r) \phi_n r d x = \int_a^b f \phi_n d x 
$$



\end{frame}


\begin{frame}{Alternativa de Fredholm}

Entonces  

$$\sum_{n=1}^{\infty}\left(\mu-\lambda_n\right) c_n \phi_n =\sum_{n=1}^{\infty} \gamma_n \phi_n$$

Por la unicidad del desarrollo en serie
$$\left(\mu-\lambda_n\right) c_n=\gamma_n.
$$

\begin{enumerate}
 \item<+-> Si $\mu \neq \lambda_n$ para cada $n=1,2,3, \ldots$, la solución es
 \begin{empheq}[box=\tcbhighmath]{equation*}
\phi=\sum_{n=1}^{\infty} \frac{\gamma_n}{\mu-\lambda_n} \phi_n.
\end{empheq}
 \item<+-> Si $\mu=\lambda_N$ para algún $N$, cuando  $n=N$ vemos que para que haya solución es necesario que
\begin{empheq}[box=\tcbhighmath]{equation*}
    0=\gamma_N=\int_a^b f(x) \phi_N(x) d x .
\end{empheq}


\end{enumerate}

 

\end{frame}
   


\begin{frame}{Alternativa de Fredholm}
 
  \emph{La función $f$ debe ser ortogonal a $\phi_N$}. Este resultado es parte de uno más general llamado \emph{alternativa de Fredholm}.
 
 El coeficiente $C_N$ se puede elegir arbitratriamente 
 y la solución es
 
  \begin{empheq}[box=\tcbhighmath]{equation*}
\phi=C_n\phi_N+\sum_{n\neq N} \frac{\gamma_n}{\mu-\lambda_n} \phi_n.
\end{empheq}
 No hay solución única, sino una familia uniparamétrica de soluciones.
\end{frame}


\section{Funciones de Green}





\subsection{La ``función'' delta, motivación}

\begin{frame}{La ``función'' delta, motivación  }

\textbf{Razonamiento heurístico.}  Supongamos una partícula moviendose sobre una recta, $x(t)$ su posición, $v(t)$ su velocidad y $a(t)$ su aceleración. Entonces 

$$\Delta v=v(t+\Delta t)-v(t)=\int_t^{t+\Delta t}a(s)ds.$$

La aceleración es una acción distribuida en el tiempo. Si ponemos por ejemplo 

$$
a(t)=\left\{ 
        \begin{array}{cc}
         2,  &\text{ si } t=0,\\
         0,  &\text{ si } t\neq 0,\\ 
        \end{array}
      \right.   
$$

Ocurrirá que $\Delta v=0$ de todas maneras. Una aceleración aplicada en un instante no produce cambios.



\end{frame}


\begin{frame}{La ``función'' delta, motivación }
No obstante es útil contar con objetos matemáticos que den cuenta de cambios grandes en un instante. Supongamos que la velocidad se incremente en 1 unidad sólo en $t=0$ ¿Qué propiedades debería tener tal aceleración? La llamaremos $\delta$. Si ponemos $t=\alpha$ y $t+\Delta t=\beta$

  $$\Delta v=v(b)-v(a)=\int_{\alpha}^{\beta}\delta(s)ds
  =\left\{ 
        \begin{array}{cc}
         1,  &\text{ si }  0\in [\alpha,\beta]\\
         0,  &\text{ si }  0\notin  [\alpha,\beta],\\ 
        \end{array}
      \right.   
.$$

Vamos a suponer que una tal $\delta$ existe e inferiremos algunas propiedades

\end{frame}




\begin{frame}{La ``función'' delta, propiedades }
 
\begin{block}{``Teorema``}
 Si $\varphi:\mathbb{R}\to\mathbb{R}$ es continua y acotada, entoces
 \[
  \int_{-\infty}^{\infty}\varphi(t)\delta(t)dt=\varphi(0).
 \]

\end{block}

 \textbf{''Demostración.''} Sea $\varepsilon>0$  y tomemos $\delta>0$  tal que 
 \[
  |t|<\delta\Rightarrow |\varphi(t)-\varphi(0)|<\varepsilon.
 \]
 Supongamos  $|\varphi|\leq M$. Entonces 
  \begin{multline*}
  \left|\int_{-\infty}^{\infty}\varphi(t)\delta(t)dt-\varphi(0)\right|=
  \left|\int_{-\infty}^{\infty}\varphi(t)\delta(t)dt-\varphi(0)\int_{-\infty}^{\infty}\delta(s)ds\right|\\
  \leq  \int_{-\infty}^{\infty}\left|\varphi(t)-\varphi(0)\right| \delta(t)dt\leq 
  \int_{|t|>\delta} 2M\delta(t)dt+\varepsilon\int_{|t|<\delta}\delta(t)dt\leq\varepsilon
   \end{multline*}
Haciendo $\varepsilon\to 0$ obtenemos la conclusión\qed
  
 

\end{frame}




\begin{frame}{La ``función'' delta, propiedades  }

  \begin{block}{``Corolario``}
 Si $\varphi:\mathbb{R}\to\mathbb{R}$ es continua y acotada, entoces
 \[
  \int_{-\infty}^{\infty}\varphi(t)\delta(x-t)dt=\varphi(x).
 \]
 
 Consecuentemente
 \begin{equation}\label{eq:prop_delta}
 \varphi(t) \delta(x-t)=\varphi(x) \delta(x-t)
 \end{equation}


\end{block}




\end{frame}


\begin{frame}{Otros intentos de definición de delta }

Dirac:
$$
\delta(t)= \begin{cases}\infty, & t=0 \\ 0, & t \neq 0\end{cases},\quad\text{ y }\quad 
\int_{-\infty}^{\infty} \delta(t) d t=1
$$
Kirchoff: 
$$
\delta(t)=\lim _{n \rightarrow \infty} \frac{n}{\sqrt{\pi}} e^{-n^2 t^2}
$$
Heaviside:
$$
\delta(t)=\frac{d H(t)}{d t} ,\quad\text{donde } H(t)= \begin{cases}1, & t>0 \\ 0, & t<0\end{cases}
$$

L. Schwartz. Para la matemática actual $\delta$ es una medida un objeto dentro del conjunto de las funciones generalizadas o distribuciones.
\end{frame}


 

\subsection{Definición }
\begin{frame}{Función de Green definición}


\begin{block}{Definición [Función de Green]}
 Se llama \emph{función de Green (o función de Green de dos puntos)}  del  problema inhomogéneo \eqref{eq:no-homogeneo} 
a la solución $G(x, s)$ correspondiente a un término inhomogéneo impulsivo:

\begin{empheq}[box=\tcbhighmath,left=\left\{,right=\right.]{equation}\label{eq:no-homogeneo}
    \begin{split}
        L_{\mu}[G]:=L[G]+\mu r G&=\delta(x-s)\\
        a_1 G(a, s)+a_2 G_x(a, s)&=0,\\ 
        b_1 G(b, s)+b_2 G_x(b, s)&=0
    \end{split}
\end{empheq}

\end{block}


\begin{block}{Ejercicio} Si $\lambda_n$ y $\phi_n$ son respectivamente la sucesión de autovalores de $L$ y sus correspondientes autofunciones ortonormales, y si $\mu\neq\lambda_n$, $n=1,2,\ldots$, entonces 
$$
G(x, s)=\sum_{n=1}^{\infty} \frac{\phi_n(x) \phi_n(s)}{\lambda-\lambda_n} .
$$
\end{block}
\end{frame}

\subsection{ Propiedades}
\begin{frame}{Solución del problema no homogéneo }

\begin{block}{Corolario}  $G(x, s)=G(s, x)$.
 \end{block}
 
\begin{block}{Corolario}  Si $\lambda$ no es autovalor de $L$ la única solución del problema inhomogéneo \eqref{eq:no-homogeneo} es, para cualquier $f(x)$,
$$
y(x)=\int_a^b G(x, s) f(s) d s,
$$
\end{block}

\textbf{Demostración.} Como $B[u,a]$ es lineal en $u$:
\[
 B\left[ y,a\right]=\int_a^b B\left[G(x, s),a\right] f(s) d s=0.
\]
Lo mismo se hace con $B[u,b]$. 
\end{frame}


\begin{frame}{Solución del problema no homogéneo }
Por la linealidad del operador $L_\mu$ tenemos:
$$
L_{\mu}[y](x)=\int_a^b L_{\mu}[G]f(s) d s=\int_a^b \delta(x-s) f(s) d s=f(x) .
$$

El razonamiento tiene algunos pasos que demandarían una justificación mejor
\end{frame}


\begin{frame}{Solución del problema no homogéneo }
Otro razonamiento también incompleto pero esclarecedor desde otro ángulo.

Tomamos una partición $P$ de $[a,b]$, $a=t_0<t_1<\ldots < t_n=b$ y aproximamos $f\approx f_P$ donde  
$$f_P(t)= f(t_i),\quad \text{cuando } t\in [t_i,t_{i+1}].$$

Si $H$ es la función de Heaveside

\begin{multline*}
 f_P(t)= \sum_{i=0}^{n-1}f(t_i) (H(t-t_{i})-H(t-t_{i+1}))\approx\sum_{i=0}^{n-1}f(t_i)\left.\frac{dH}{ds}\right|_{s=t-t_i}(t_{i+1}-t_i)\\
 =\sum_{i=0}^{n-1}f(t_i)\delta(t-t_i)(t_{i+1}-t_i)=\sum_{i=0}^{n-1}f(t_i)L_{\mu}[G(t,t_i)](t_{i+1}-t_i)\\
 =L_{\mu}\left[ \sum_{i=0}^{n-1}f(t_i)G(t,t_i)(t_{i+1}-t_i)  \right]\approx L_{\mu}[y]
\end{multline*}
 


\end{frame}


\begin{frame}{Ejemplo función de Green}

\begin{block}{Ejercicio} Demostrar que la función de Green de 
 $$
y^{\prime \prime}+\lambda y=f(x), \quad y(0)=y(\ell)=0,
$$
es 
$$
G_\lambda(x, s)=\frac{2}{\ell} \sum_{n=1}^{\infty} \frac{\sin n \omega x \sin n \omega s}{\lambda-n^2 \omega^2},\quad\omega=\frac{\pi}{\ell} .
$$
\end{block}

\end{frame}

\subsection{Caracterización}
\begin{frame}{Función de Green, caracterización}

\begin{block}{Teorema} Supongamos que  $\mu$  no es un valor propio del correspondiente problema homogéneo de \eqref{eq:no-homogeneo}. Sean  $y_1$ e $y_2$ dos soluciones del problema homogéneo, tales que cada una de ellas satisface una de las dos condiciones de contorno (pero no la otra),
$$
\begin{array}{lll}
    L_\mu[ y_1]=0, & a_1 y_1(a)+a_2 y_1^{\prime}(a)=0, & b_1 y_1(b)+b_2 y_1^{\prime}(b) \neq 0 \\ 
    L_\mu[ y_2]=0, & b_1 y_2(b)+b_2 y_2^{\prime}(b)=0, & a_1 y_2(a)+a_2 y_2^{\prime}(a) \neq 0
\end{array}
$$
\end{block}
\end{frame}


\begin{frame}{Función de Green, caracterización}

\begin{block}{Teorema (contiuación)}

\begin{enumerate}
 \item<+-> $y_1$ e $y_2$ son linealmente independientes, $W(x)=W\left[y_1, y_2\right] \neq 0$.
 
 \item<+-> $p(x) W(x)$ es constante.

 \item<+-> La función de Green del problema es
$$
G(x, s)= \begin{cases}\frac{y_1(x) y_2(s)}{p(s) W(s)}, & \text { para } a \leq x \leq s, \\ \frac{y_1(s) y_2(x)}{p(s) W(s)}, & \text { para } s \leq x \leq b .\end{cases}
$$

\item<+-> La función de Green es continua, pero su derivada tiene un salto de valor $1 / p(s)$ en $x=s$ :
$$
G_\lambda(s+0, s)=G_\lambda(s-0, s), \quad G_\lambda^{\prime}(s+0, s)-G_\lambda^{\prime}(s-0, s)=\frac{1}{p(s)} .
$$
\end{enumerate}

\end{block}


\end{frame}


\begin{frame}{Demostración caracterización}

\emph{1) $y_1$ e $y_2$ son linealmente independientes} Si $y_1=ky_2$, $y_1$ satisfacería ambas condiciones y esto contradice  que $\mu$ no es un valor propio.

\emph{ 2) $p(x) W(x)$ es constante} 

 
\begin{multline*}
{[p W]^{\prime} } =\left[p\left(y_1 y_2^{\prime}-y_1^{\prime} y_2\right)\right]^{\prime}=  y_1\left(py_2^{\prime}\right)^{\prime}-y_2\left(p y_1^{\prime}\right)^{\prime}\\
=-y_1(q+\mu r) y_2+y_2(q+\mu r) y_1=0 .
\end{multline*}
 
\emph{4) $G$ es continua.} Sigue de la definición que:
$$
G (s+0, s)=G_\lambda(s-0, s)=\frac{y_1(s) y_2(s)}{p(s) W(s)} .
$$

Luego $G$ es continua.
\end{frame}


\begin{frame}{Demostración caracterización}
\emph{4) $G_x$ tiene un salto.}
Para la derivada,

$$
G_x(x, s)= \begin{cases}\frac{y_1'(x) y_2(s)}{p(s) W(s)}, & \text { para } a \leq x \leq s, \\ \frac{y_1(s) y_2'(x)}{p(s) W(s)}, & \text { para } s \leq x \leq b .\end{cases}
$$

se obtiene
$$
G_\lambda^{\prime}(s+0, s)-G_\lambda^{\prime}(s-0, s)=\frac{y_1(s) y_2^{\prime}(s)-y_1^{\prime}(s) y_2(s)}{p(s) W(s)}=\frac{1}{p(s)}
$$

\emph{3) fórmula.} Se puede escribir usando $H$ función de Heaveside
$$G(x,s)= \frac{y_1(x) y_2(s) H(s-x)+y_1(s) y_2(x) H(x-s)}{p(s) W(s)}$$
$$G_x(x,s)= \frac{y_1^{\prime}(x) y_2(s) H(s-x)+y_1(s) y_2^{\prime}(x) H(x-s)}{p(s) W(s)}$$



\end{frame}


\begin{frame}{Demostración caracterización}
Para la segunda derivada  
 
\begin{multline*}
G_{xx}(x, s)  =\frac{y_1^{\prime \prime}(x) y_2(s) H(s-x)+y_1(s) y_2^{\prime \prime}(x) H(x-s)}{p(s) W(s)} \\
+\frac{\left[y_1(s) y_2^{\prime}(x)-y_1^{\prime}(x) y_2(s)\right] \delta(x-s)}{p(s) W(s)} \\
 =\frac{y_1^{\prime \prime}(x) y_2(s) H(s-x)+y_1(s) y_2^{\prime \prime}(x) H(x-s)}{p(s) W(s)}\\
 +\frac{y_1(x) y_2^{\prime}(x)-y_1^{\prime}(x) y_2(x)}{p(x) W(x)} \delta(x-s) \\
 =\frac{y_1^{\prime \prime}(x) y_2(s) H(s-x)+y_1(s) y_2^{\prime \prime}(x) H(x-s)}{p(x) W(x)}+\frac{1}{p(x)} \delta(x-s)
\end{multline*}
 
\end{frame}


\begin{frame}{Demostración caracterización}
Entonces
\begin{multline*}
 L_\mu[G(x, s)]=p(x)G_{xx}+p'(x)G_x+(q+\mu r )G\\
 \delta(x-s)+\frac{p(x)y_1^{\prime \prime}(x) y_2(s) H(s-x)+p(x)y_1(s) y_2^{\prime \prime}(x) H(x-s)}{p(s)W(s)}\\
 +\frac{p'(x)y_1^{\prime}(x) y_2(s) H(s-x)+p'(x)y_1(s) y_2^{\prime}(x) H(x-s)}{p(s) W(s)}\\
 +\frac{(q+\mu r )y_1(x) y_2(s) H(s-x)+(q+\mu r )y_1(s) y_2(x) H(x-s)}{p(s) W(s)}\\
 = \delta(x-s)
 \end{multline*}
 
 
  
Es facil ver que  se cumplen las condiciones de contorno \qed

\end{frame}


\begin{frame}{Ejemplo}

\begin{block}{Ejercicio} Demostrar que cuando $\mu=0$ en el ejercicio anterior
$$
G_\lambda(x, s)= \begin{cases}\frac{x(s-\ell)}{\ell}, & \text { para } 0 \leq x \leq s \\ \frac{s(x-\ell)}{\ell}, & \text { para } s \leq x \leq \ell\end{cases}
$$

\emph{Fórmulas para $\pi$} Comparando con la representación previa de $G$ deducir 
\[
\begin{split}
 \sum_{n=1}^\infty\frac{1}{(2n-1)^2}&=\frac{\pi^2}{8},\\
  \sum_{n=0}^\infty(-1)^n\frac{1}{2n+1}&=\frac{\pi}{4}.
\end{split}
\]

\end{block}






\end{frame}
\end{document}
